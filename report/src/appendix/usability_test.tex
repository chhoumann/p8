\section{Walkthrough- and understandability tasks}\label{appendix:usability_tasks}
\underline{\textbf{Information til første del - Struktureret interview:}}
Den første halvdel af undersøgelsen handler om at vi skal finde ud af, hvor godt designet viser information. Derfor skal du ikke interagere med det vi viser dig. Du skal blot svare på de spørgsmål vi stiller i henhold til det opstillede scenarie.
\newline\newline
\underline{\textbf{Scenarie 1:}}
Forestil dig at du er på gangen og kommer i tanke om at du har brug for at booke et mødelokalet mellem 14:30 og 15:00.
Derfor går du hen til nærmeste mødelokale. Her lægger du mærke til at tabletten foran lokalet viser noget på skærmen.
\newline\newline
**Præsentér rødt UI**
\newline\newline
\textbf{Spørgsmål 1:}
Er der på nuværende tidspunkt et møde igang i lokalet?
\newline
\textbf{Spørgsmål 2:}
Hvem har booket det igangværende møde?
\newline
\textbf{Spørgsmål 3:}
Er lokalet ledigt mellem 14:30 og 15:00?
\newline\newline
\underline{\textbf{Scenarie 2 - Observations studie:}}
Du konkludere at lokalet ikke er tilgængeligt i det tidsrum du skal holde møde. Derfor går du hen til nabolokalet, som også er et mødelokale.  
\newline\newline
**Vis grønne UI**
\newline\newline
\textbf{Spørgsmål 1:}
Er dette lokale ledigt lige nu?
\newline
\textbf{Spørgsmål 2:}
Hvornår er det næste møde i dette lokale?
\newline
\textbf{Spørgsmål 3:}
Er det muligt at booke lokalet mellem kl. 14:30 og klokken 15:00?
\newline\newline
\underline{\textbf{Information til anden halvdel:}}
Vi går nu videre til anden halvdel af undersøgelsen. Denne halvdel handler om, at vi skal finde ud af, om det design vi har er nemt at interagere med. Derfor får du nu et par små opgaver du skal løse. 
Husk! Da vi arbejder med en prototype er det ikke alle ’knapper’ der virker. 
\newline\newline
**Vis grønt UI**
\newline\newline
\underline{\textbf{Opgave 1:}}
Du vil gerne booke et møde, som skal afholdes med Lars Østergaard:
\newline
\textbf{Book lokalet mellem 14:30 og 15:00. Inviter Lars Østergaard til mødet.}
\newline\newline
**Vis rødt UI**
\newline\newline
\underline{\textbf{Opgave 2:}}
Mødet er nu tæt på at være færdigt. Du har brug for yderligere en halv time til at færdiggøre mødet.  Derfor vil du gerne udvide mødet med 30 minutter. 
\newline
\textbf{Din opgave er at udvide mødet med 30 minutter. }
\newline\newline
**Vis rødt UI**
\newline\newline
\underline{\textbf{Opgave 3:}}
Det viste sig at du kun havde brug for 15 minutter til at færdiggøre mødet. Derfor vil du gerne registerere at mødet er sluttet tidligt så andre kan begynde at bruge lokalet.
\newline
\textbf{Afslut det nuværende møde. }