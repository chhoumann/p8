\section{Choice of technique}\label{sec:ChoiceOfTechnique}
When picking a technique for determining whether a room is occupied, we consider several criteria.
The technique must not compromise the technology criteria presented in Table \ref{lst:presence_eval_criteria}.
Additionally, a system used for presence detection does not have to estimate the exact position of objects.
It only needs to be accurate enough for the system to classify the object as present or not present.
In the case of this project, a high level of accuracy and precision is not necessary, as occupancy detection does not rely on exact positioning.
As occupants leave the room, the system does not need to immediately assert whether the room is occupied or not.
The assertion can be made once the system is reasonably certain that all occupants have left the near vicinity of the meeting room.
Therefore, the chosen technique should be able to support the classification of object positions.  
For this purpose, lateration techniques could utilize a distance measure, and classify everything with an estimated distance smaller than some threshold as being present.
Variations of Equation \ref{distance_equation} could be examined to determine which approach is best for the system.
However, lateration approaches rely on a non-noisy environment, as they rely directly on the measured RSSI values.
Fingerprinting techniques could use the grid measurements, established in the first phase of the approach, described in Section \ref{sec:fingerprinting}, with additional labels for each cell, determining whether the cell is inside or outside the meeting room.
This would create a set of labelled data, where the labels of the k-nearest neighbors could be used to determine whether an object is present. 

Between the two approaches, the fingerprinting approach can easily be configured with beacons to increase accuracy.
Edge cases for this approach can be analyzed and handled by setting some threshold for when an \textit{inside} label is considered correct. 
This threshold could be found experimentally or determined from existing research. 
For instance, \citeauthor{ble_kneares_neural}~\cite{ble_kneares_neural} showed that a kNN approach yielded high precision, in a range of up to 3 meters.
As the goal of the project is not to localize people in a room, but to classify whether they are present or not, we will use the kNN method for classification.
Additionally, kNN methods are comparatively cost-effective in terms of computation, which is crucial given that classification tasks will be executed frequently.
This precision could be used as a threshold to determine to modify which neighbors to consider.    
