% \section{Approach}\label{sec:approach}
% - We use fingerprinting.
% - Phase 1: Data collection in grid
%     - Collect data by measuring RSSI values at different locations in the room.
%     - The locations are shaped by the grid.
% 	- Each measurement is made up of one (averaged over samples in that location) RSSI value per beacon, along with a label designating whether the grid tile represents being inside or outside the room.
% - Phase 2: Classification
%     - We use kNN to classify the data into either \textit{inside} or \textit{outside}.
%     - The kNN algorithm finds the k-nearest neighbors of the data for the latest measurement.
%     - We weigh the nearest neighbors by their distance from the latest measurement (close neighbors weigh more). The resulting weight is calculated as the (confidence? certainty?) of being inside the room.
%     - We then label the data as either \textit{inside} or \textit{outside}.

\section{Our Approach}\label{sec:our_approach}
Based on our prior discussion in Section \ref{sec:ChoiceOfTechnique} on potential approaches, we have opted to use a fingerprinting technique for our system.
Our methodology is split into two main phases: data collection and classification.
Each phase plays a critical role in successfully implementing this technique and, when combined, they offer an effective method of presence detection.

\subsection{Phase 1: Data Collection}\label{sec:phase1_data_collection}
In the first phase, our focus is on collecting RSSI values from various locations within a predefined area. The area is partitioned into a grid, with each cell in the grid representing a specific location. 

For each cell of the grid, we measure and record the RSSI value from every beacon within the system.
The RSSI value is averaged over several samples to account for fluctations.
Furthermore, we augment each measurement with a label indicating whether the corresponding grid cell is considered to be \textit{inside} or \textit{outside} the room.
This labeling is necessary for the classification stage of our approach.
In the context of the system, we refer to the collected measurements as the \textit{data set}.

\subsection{Phase 2: Classification}\label{sec:phase2_classification}
In the second phase, we employ a k-Nearest Neighbors algorithm to classify our collected data. Our labels from the data collection phase serve as the classes: \textit{inside} and \textit{outside}.

By continuously listening for beacon advertisements, we store the latest RSSI values in memory. 
We then apply the kNN algorithm to identify the k-nearest values to the most recent RSSI readings.

Using the latest measurement, the kNN algorithm identifies the k-nearest neighbors from the data set. 
The nearest neighbors are found based on their Euclidean distances to the latest measurement.

We assign each neighbor a weight that is inversely proportional to its distance from the new measurement, implying that neighbors closer to the measurement have a larger influence on the classification decision.

The final classification is determined by summing the weights of the neighbors belonging to each class, and checking whether this sum exceeds some preset threshold.
The resulting weight is the confidence of the presence within the room.



The quality of the classification depends on the threshold.
Therefore, it is important to tune the threshold to suit the environment in order to provide a good compromise between accuracy and precision.

Hence, the confidence of the presence within the room is derived from the total weight of the \textit{inside} class neighbors.

This two-phase approach, combining data collection via grid partitioning and kNN classification, forms the basis of our fingerprinting-based presence detection system.
The following sections delve into the key design decisions and the specifics of implementing this approach.