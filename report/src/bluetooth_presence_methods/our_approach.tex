\section{Our Approach}\label{sec:our_approach}
Based on our prior discussion in Section \ref{sec:ChoiceOfTechnique} on potential approaches, we have opted to use a fingerprinting technique for our system.
Our methodology is split into two main phases: data collection and classification.
Each phase plays a critical role in successfully implementing this technique and, when combined, they offer an effective method of presence detection.

\subsection{Phase One: Data Collection}\label{sec:phase1_data_collection}
In the first phase, our focus is on collecting RSSI values from various locations within a predefined area. The area is partitioned into a grid, with each cell in the grid representing a specific location. 

For each cell of the grid, we measure and record the RSSI value from every beacon within the system.
In each cell of the grid, we take measurements and store the Received Signal Strength Indicator (RSSI) values from every beacon that is part of the system. 
We average the RSSI values over multiple samples to mitigate the effect of any fluctuations, thereby obtaining a reliable final data point. 
In the data collection process, a single receiver device is manually moved to each grid point sequentially, ensuring that necessary data is collected from every location.
We only secure each sample once we have confirmed the RSSI values have reached a stable state. This stability is determined using the standard deviation.
Upon achieving this stability, we store the averaged RSSI value.
Each sample includes a label indicating whether the corresponding grid cell is considered to be \textit{inside} or \textit{outside} the room.
The labeling is necessary for the classification stage of our approach.
In the context of the system, we refer to the collected samples as the \textit{data set}.
This data set represents the map of the given room, as described in Section \ref{sec:fingerprinting}.


\subsection{Phase Two: Classification}\label{sec:phase2_classification}
In the second phase, we employ a k-Nearest Neighbors algorithm to classify our collected data. Our labels from the data collection phase serve as the classes: \textit{inside} and \textit{outside}.

By continuously listening for beacon advertisements, we store the latest RSSI values in memory. 
We then apply the kNN algorithm to identify the k-nearest values to the most recent RSSI readings.

Using the latest measurement, the kNN algorithm identifies the k-nearest neighbors from the data set. 
The nearest neighbors are found based on their Euclidean distances to the latest measurement.

We assign each neighbor a weight that is inversely proportional to its distance from the new measurement, implying that neighbors closer to the measurement have a larger influence on the classification decision.
This computation is expressed in equation \ref{eq:calculate_weight}.

\begin{equation}\label{eq:calculate_weight}
W_j = \sum_{i=1}^{k} \frac{I(y_i = j)}{d_i} \Bigg/ \sum_{i=1}^{k} \frac{1}{d_i}
\end{equation}

where $k$ is the number of nearest neighbors, $y_i$ is the class label of the $i$-th neighbor, $d_i$ is the distance from the test point to the $i$-th neighbor, $I(y_i=j)$ is the indicator function, which is $1$ when $y_i=j$, and $0$ otherwise, and $W_j$ is the calculated weight for class $j$.

For each neighbor in the k-nearest-neighbors, it checks if the neighbors class label is equivalent to the target class label. As shown by the indicator function in equation~\ref{eq:calculate_weight}, we only consider the inverse distance of neighbors with the target class label. This is done to ensure that neighbors with the opposite class label do not contribute to the sum.

The resulting weight $W_j$ indicates the likelihood of the latest measurement belonging to the given class label.

We compare $W_j$ to some preset threshold $t$. Therefore, the classification decision $D$ is determined by:

$$
D = 
\begin{cases}
\text{"inside"} & \text{if } W_j \geq t \\
\text{"outside"} & \text{if } W_j < t 
\end{cases}
$$

Here, $D$ is the classification decision, which takes the value \textit{inside} if $W_j$ is greater than or equal to the threshold $t$, and \textit{outside} if $W_j$ is less than $t$.

The quality of the classification depends on the threshold $t$.
Therefore, it is important to tune the threshold to suit the environment in order to provide a good compromise between accuracy and precision.

This two-phase approach, combining data collection via grid partitioning and kNN classification, forms the basis of our fingerprinting-based presence detection system.
The following sections delve into the key design decisions and the specifics of implementing this approach.