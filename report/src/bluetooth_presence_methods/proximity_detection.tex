\section{Proximity Detection}\label{sec:proximity_detection}
Proximity detection computes distances using the Received Signal Strength Indicator (RSSI).
Utilizing RSSI values, the proximity of devices, such as smartphones or wearables, to a central hub in the meeting room can be determined. 
This determination is based on the assumption that as the signal strength diminishes, the distance between the device and the hub increases. 
Although the signal strength may fluctuate due to environmental factors, this method provides a rudimentary approximation of the user's location.\cite{spachosBLEBeaconsIndoor2020}

The principle of proximity detection is fundamentally rooted in the path-loss model, which correlates the received signal strength with the distance from the transmitter. This relationship can be formulated mathematically as follows:\cite{spachosBLEBeaconsIndoor2020}
$$
d_{noiseless} = d_0 \cdot 10^{ \frac{(RSSI_{0} - RSSI)}{10 \cdot n}}
$$

In this equation, \(d_{noiseless}\) symbolizes the estimated distance in an ideal, noiseless environment expressed in meters.
$d_0$ is the reference distance measured in meters, while \(RSSI_{0}\) represents the transmitted signal strength distance $d_0$ to the transmitter, and \(RSSI\) denotes the received signal strength, both expressed in decibels-milliwatts (dBm). 
Due to the assumption of the transmitted signal strength diminishing, the expression $(RSSI_{0} - RSSI) \geq 0\ dBm$ holds. 
\(n\) denotes the path loss component, which expresses the rate of signal decay with distance.  
The path loss component \(n\) is chosen based on environmental factors \cite{spachosBLEBeaconsIndoor2020}. 

Despite its merits, including cost-effectiveness, ease of implementation, and scalability, proximity detection has significant limitations in terms of accuracy, primarily due to the potential for inaccurate range estimation. 
The intrinsic nature of Bluetooth signals causes the signal strength to fluctuate based on environment factors, such as obstructions or radio-wave noise.\cite{spachosBLEBeaconsIndoor2020} 
Furthermore, the environmental component $n$ in the formula must be adjusted to account for these environmental factors.  
This implies that any system utilzing this approach must be calibrated for it to yield accurate results in a particular environment. 
Such an undertaking can be labor-intensive and may necessitate periodic recalibration.\cite{spachosBLEBeaconsIndoor2020}

To summarize, while proximity-based presence detection using RSSI offers a foundational and relatively easy-to-implement mechanism for automating meeting room occupancy management, the concerns regarding its accuracy present a significant limitation. 
Considering alternative methodologies with higher accuracy could lead to the establishment of a more robust and reliable solution for managing meeting room occupancy.
