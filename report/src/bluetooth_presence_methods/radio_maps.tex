\section{Fingerprinting}
Fingerprinting is a technique utilizing the measured power of received bluetooth signal to accomodate the physical layout of the room in which the system is deployed. 
Setting up a system utilizing fingerprinting requires two phases: a phase where the layout of the room is analysed to accomodate impact of physical obstacles in the room, and a phase where the unknown position of a bluetooth receiver is inferred \cite{presence_ble_review, taking_localization_to_the_wild}. Often, this map used to create machine-learning models \cite{taking_localization_to_the_wild}.

\citeauthor{taking_localization_to_the_wild} \cite{taking_localization_to_the_wild} presents results for a fingerprinting approach using random forrest regression.
They show how varying measured RSSI accross devices can be mitigated by using RSSI difference between pairs of beacons.
They conclude that maps created by the measuring device in the first phase can reused to create maps for other types of devices without significant loss of accuracy. 
\citeauthor{presence_ble_review}\cite{presence_ble_review} presents several ways to use the map to create models for fingerprinting, but does not evaluate them.
\citeauthor{improving_indoor_localization} \cite{improving_indoor_localization} choose to use a K-nearest neighbor directly on the observed data, and compute the average  location between the selected k neighbors and the actual positione. 
Their experiments is performed in area of $52x43$ meters using 17 BLE beacons.
The experiment yields accurate results, although with a few drastic outliers. 
They conclude that the number of beacons greatly affects the accuracy of the approach. 


