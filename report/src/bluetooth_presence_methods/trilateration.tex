\section{Lateration}
Lateration is a technique using multiple, immovable reference points to estimate the distance to an object~\cite{presence_ble_review}.

During the technique, the distance from the subject of interest to multiple nodes must be known. 
As both the location and an approximate distance to the nodes are known, one can estimate the location of the subject by examining circles defined by the distance from the subject to the nodes.
The point of intersection of these circles will define the approximate location of the subject.\cite{laterationExplanation}  

In the context of Bluetooth systems, the nodes are often BLE beacons that broadcast information about their specification. 
The content of this broadcast is called an \textit{adverticement}. 
We will refer to the nodes as \textit{broadcasters} while any entity listening and reacting to its adverticements as a \textit{receiver}.
When using Bluetooth, the reference points are often BLE beacons and the receiver some other Bluetooth-capable device.~\cite{apple2023ibeacon} 

The beacons periodically broadcast an advertisement, notifying receivers about information about themselves. 
If the beacon uses the iBeacon protocol ~\cite{apple2023ibeacon} the adverticement will include information about the beacon's transmission power, its name, and a group used to categorize the device.
The advertisement and its content are collected by receivers which uses the signal strength of the received signal and the information about the beacon's transmission power to calculate distance estimations to each beacon. 

By analysing the distance estimations between the beacons and the receiver, the position of the receiver is estimated.
Various techniques exists to determine this distance.
\citeauthor{presence_ble_review}~\cite{presence_ble_review} examines five techniques, each using different combinations of advertisement content, to establish a distance measure to each beacon, and present current methods for handling the loss of signal during the analysis.

Trilateration using Bluetooth is a form of lateration using the known transmission power of one or more beacons to calculate a \textit{path loss}.
Path loss defines the amount of signal power lost as the signal is sent across a distance of $d$ meters.
As the transmission power of the beacons are known, the loss per distance $L(d)$ can be subtracted from the transmission power $p_0$ to find the distance~\cite{taking_localization_to_the_wild}:
\begin{equation}\label{distance_equation}
    f(d) = p_0 - L(d)
\end{equation}
The path loss is affected by background signal noise and the air in the atmosphere.
This can be defined as $L(d) = 10n log_{10}(d)+C$, where $C$ represents signal loss and background noise, and $n$ is a loss factor. \cite{taking_localization_to_the_wild}
Various models using different variants of constants $C$ and $n$ exist\cite{path_loss_models}.
Due to the many possible environmental circumstances in which the beacons and receiver may reside, the accuracy of the approach may be flawed\cite{presence_ble_review}. 
Multiple mathematical models of the path loss formula exist \cite{rssi_indoor_pos,positioning_alg_rssi, RSSI_ZigBee_distance}, making it trivial to implement this approach for location approximations.

Once the distance for the path loss to each beacon is computed, using one of these approaches, the path loss can be used to establish spheres with a radius equal to the the distance from the receiver to each of the broadcasters. 
The intersection of these spheres will define the approximate location of the receiver. 