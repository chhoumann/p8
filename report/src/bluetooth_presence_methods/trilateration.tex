\section{Lateration}
Lateration is a technique using multiple, immuvable reference points to estimate the distance to an object\cite{presence_ble_review}.
We will refer to this object as the \textit{receiver}   
When using bluetooth, the reference points are BLE beacons and the receiver some bluetooth device. 
The beacons periodically broadcast a signal adverticing their existence. 
The adverticement and its content is collected by the receiver, which uses it to calculate distance estimations to each beacon. 
By analysing the distance estimations, the position of the receiver is determined.
Various versions of the techniques exists.
\citeauthor{presence_ble_review}~\cite{presence_ble_review} examines five techniques each using different combinations of adverticement content to establish a distance measure, and present current methods for handling loss of signal.

Trilateration is a form of lateration using the transmission power of the beacon to calculate a \textit{power path loss}(PPL).
PPL $L(d)$ defines the amount of power lost as the signal is sent across a distances of $d$ meters.
As the transmission power of the beacons are known, the loss per distance $L(d)$ can be subtracted from the transmission power to find the distance:\cite{taking_localization_to_the_wild}
\begin{equation}
    f(d) = p_0 - L(D)
\end{equation}

The determinition of $L(D)$ 
