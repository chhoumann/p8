\section{Lateration}
Lateration is a technique using multiple, immovable reference points to estimate the distance to an object\cite{presence_ble_review}.
We will refer to this object as the \textit{receiver}.
When using Bluetooth, the reference points are often BLE beacons and the receiver some other Bluetooth-capable device. 
The beacons periodically broadcast an advertisement, notifying about their existence.\cite{apple2023ibeacon} 
The advertisement and its content are collected by the receiver, which uses it to calculate distance estimations to each beacon. 
The content of the advertisement often contains information about the beacon device. 
For one of the more common protocols, the information includes the transmission power of the device\cite{apple2023ibeacon}.

By analysing the distance estimations between the beacons and the receiver, the position of the receiver is estimated.
Various techniques exists to determine this position.
\citeauthor{presence_ble_review}~\cite{presence_ble_review} examines five techniques, each using different combinations of advertisement content, to establish a distance measure, and present current methods for handling the loss of signal during the analysis.

Trilateration is a form of lateration using the transmission of one or more beacons to calculate a \textit{path loss}.
Path loss defines the amount of signal power lost as the signal is sent across a distance of $d$ meters.
As the transmission power of the beacons are known, the loss per distance $L(d)$ can be subtracted from the transmission power $p_0$ to find the distance~\cite{taking_localization_to_the_wild}:
\begin{equation}
    f(d) = p_0 - L(D)
\end{equation}
The path loss is affected by background signal noise and the air in the atmosphere.
This can be defined as $L(d) = 10n log_{10}+C$, where $C$ represents signal loss and background noise, and $n$ is a loss factor. \cite{presence_ble_review}
Various models using different variants of constants $C$ and $n$ exist\cite{path_loss_models}.
Due to the many possible environmental circumstances in which the beacons and receiver may reside, the accuracy of the approach may be flawed\cite{presence_ble_review}. 
Multiple mathematical models of the path loss formula exist \cite{rssi_indoor_pos,positioning_alg_rssi, RSSI_ZigBee_distance}, making it easy to implement this approach for location approximations.