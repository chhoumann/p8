\chapter{Conclusion}
Summarizing this investigation, the report examines the complexities encountered in managing meeting spaces. 
Through a usability evaluation of a devised prototype, it was concluded that a viable solution to the identified problem could be the deployment of a presence detection system to determine the occupancy status of the stakeholder's meeting rooms.
Through an exploration of technologies suitable for presence detection in Chapter \ref{chap:presence_intro}, an extensive analysis was conducted to select an appropriate technology for such a system.
After careful consideration, Bluetooth Low Energy (BLE) emerged as the most promising technology. 
Building upon this foundation, Chapter \ref{sec:presence_detection_using_ble} delves into three distinct techniques for presence detection utilizing BLE.
Following a detailed assessment, the fingerprinting technique, using the k-nearest neighbors approach, emerged as a suitable candidate for presence detection.
Having established that this technique was a suitable candidate, we proceeded with the development of a mobile application implemented in C#. 
This application served as a facilitator for experiments used to establish whether the approach was suitable for the aforementioned system. 
The obtained results from these experiments demonstrated that our approach yielded satisfactory precision in detecting the presence of individuals within the experimental room.

This indicates that presence detection systems using BLE and fingerprinting can effectively automate the tracking of meeting room occupancy.
By minimizing the dependence on human interaction, these systems could enhance the management of meeting rooms, leading to more efficient use and, consequently, a potential reduction in administrative costs.