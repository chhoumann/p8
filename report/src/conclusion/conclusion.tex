\chapter{Conclusion}
This report examines a stakeholder's problems regarding the management of meetings. 
By performing a usability evaluation of a developed prototype, it is concluded that a solution to their problem might involve a presence detection system to establish whether the stakeholder's meeting rooms are occupied. 
Chapter \ref{chap:presence_intro} describes and evaluates technologies appropriate for presence detection, guided by a set of criteria considered necessary to address the stakeholder's problem. 
Of these technologies, Bluetooth Low Energy was found to be the best candidate. 
Three techniques for presence detection using BLE were presented and analyzed in Chapter \ref{sec:presence_detection_using_ble}. 
It was found that fingerprinting using k nearest-neighbors would be a suitable candidate for presence detection.

Subsequently, we conducted an experimental evaluation to determine if the selected technique could effectively contribute to a system aimed at resolving the stakeholder's problems. 
The development of a mobile application, implemented in C\#, was created to facilitate these experiments. 
The results of the experiments indicated that our approach could achieve satisfactory accuracy in detecting whether a person was present within the room where the experiments were carried out.

In conclusion, the findings of our investigation suggest that presence detection technology, particularly Bluetooth Low Energy, can be reliably utilized to automatically detect and indicate the status of a meeting, therefore minimizing the reliance on user interaction.