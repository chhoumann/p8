\chapter{Conclusion}
Summarizing this investigation, the report examines the complexities encountered in managing meeting spaces. 
Through a usability evaluation of a devised prototype, the research concludes that a viable solution to the identified problem could be the deployment of a presence detection system to determine the occupancy status of the stakeholder's meeting rooms.
Through an exploration of technologies suitable for presence detection, an extensive evaluation was conducted to address the stakeholder's problem (see Chapter \ref{chap:presence_intro}). 
After careful consideration, Bluetooth Low Energy (BLE) emerged as the most promising choice. 
Building upon this foundation, Chapter \ref{sec:presence_detection_using_ble} delves into three distinct techniques for presence detection utilizing BLE.
Following a detailed assessment, the fingerprinting technique, using the k-nearest neighbors approach, emerged as a suitable candidate for presence detection

Subsequently, we conducted an experimental evaluation to determine if the selected technique could effectively contribute to a system aimed at resolving the stakeholder's problems. 
The development of a mobile application, implemented in C\#, was created to facilitate these experiments. 
The results of the experiments indicated that our approach could achieve satisfactory accuracy in detecting whether a person was present within the room where the experiments were carried out.

In conclusion, the findings of our investigation suggest that presence detection technology, particularly Bluetooth Low Energy, can be reliably utilized to automatically detect and indicate the status of a meeting, therefore minimizing the reliance on user interaction.