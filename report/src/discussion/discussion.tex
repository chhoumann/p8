\chapter{Discussion}\label{chap:discussion}
The development of the project went through two major phases. 
During the first phase, the goal was to determine the needs of our stakeholder, Mariendal IT. 
To achieve this we developed both a LoFi and a HiFi prototype of a UI suitable for managing a meeting room through a tablet.
Conducting a usability test with representatives from Mariendal IT, demonstrated that our prototypes corresponded to a system that they deemed desirable and partly solve their problem.
However, during the testing process, it was discovered that an essential part of the problem was the reliance on the individuals to manage the meetings occupation of a meeting room.
Therefore, an essential part of solving their problem would be to automate this process and therefore, determining the appropiate approach to do presence detection.
% Bør man kommentere på hvor godt det var at vi lavede usability test?

In chapter \ref{chap:presence_intro} we presented five metrics that would be used to evaluate a selection of technologies for presence detection. 
Based on our research, we determined that BLE satisfied most of these metrics in relation to our stakeholders problem.
Using BLE allowed us to consider several techniques for presence detection purposes of which we considered lateration and fingerprinting as primary candidates.
Due to the need for non-noisy environments for lateration to work, we determined that the most appropiate technique would be fingerprinting along with kNN for classification.
Based on this, we implemented a solution utilizing BLE beacons and fingerprinting and in chapter x, we presented our results and our evaluation of these results. 
We found that utilizing four beacons to outline a room and a map generated by traversing the room in a grid pattern of $1m \times 1m$ we were able to achieve significant results. 
Depending on the chosen \textit{k} nearest neighbors we were able to achieve between $92.86$ and $95.71$ percent accuracy for the inside case and between $48$ and $88$ percent accuracy for the outside scenario. 
As mentioned in \ref{sec:ChoiceOfTechnique} the goal of the project was not to demonstrate high accuracy and precision on the exact positioning of persons in a room. 
The goal was rather to demonstrate that we would be able to detect the presence of people in a meeting with reasonable certainty.
Despite this objective, the results of our experimentation show that for

% Disse punkter skal være intro:
% valgt tech baseret på de 6? criteria
% Kom frem til ble var suitable
% undersøgt om den tech kan bruges til vores scenarie

% In chapter x ..

%Summary of Key Findings: Start with a brief summary of your main findings. This helps readers understand the context for the rest of the discussion.

%Interpretation of Results: Explain what the results mean in a broader context. How do they relate to the overall purpose of your study? What are their implications?

%Relating to Previous Research: Your findings should be compared with other similar studies. Do your results agree or disagree with the existing literature? If there are discrepancies, you should try to explain why this might be. If your results are unexpected or unusual, explain why this might have happened.

%Limitations: No study is perfect, and it's important to acknowledge the limitations of your research. This might include methodological limitations, such as sample size or measurement techniques, or more general limitations, such as the scope of your research or any assumptions you've made.

%Implications: Consider the practical, theoretical, and policy implications of your findings. How can they be applied in real-world contexts? How do they contribute to the theoretical understanding of your topic?

%Future Research: What questions have been left unanswered? Where could other researchers go from here? Identify areas where further research could be beneficial to build upon your study.

%Conclusion: Wrap up your discussion with a brief conclusion that summarizes the main points you've made.

%Remember, the discussion section isn't just about presenting your findings. It's about making an argument that supports your research question and explaining how your results contribute to the existing body of knowledge. Be sure to keep it clear, concise, and focused on your research.