\chapter{Discussion}\label{chap:discussion}
The project involved two main phases: initially, we designed LoFi and HiFi prototypes for a tablet-based meeting room management system, which were well-received by the stakeholder, Mariendal IT.
However, through the usability test, it became evident that automating presence detection was crucial to fully address their needs.
This led to the problem formulation in section \ref{sec:problem_statement}.

In Chapter \ref{chap:presence_intro}, we presented five metrics that would be used to evaluate a selection of technologies for presence detection.
Based on our research, we determined that BLE satisfied most of these metrics in relation to our stakeholders problem.
Using BLE allowed us to consider several techniques for presence detection purposes of which we considered lateration and fingerprinting as primary candidates.
Due to the need for non-noisy environments for lateration to work, we determined that the most appropiate technique would be fingerprinting along with kNN for classification.
Based on this, we implemented a solution utilizing BLE beacons and fingerprinting and in chapter \ref{chap:system_design}, we presented our results and our evaluation of these results. 
We found that utilizing four beacons to outline a room and a map generated by traversing the room in a grid pattern of $1m \times 1m$ we were able to achieve significant results.
Depending on the chosen \textit{k} for nearest neighbor search, we were able to achieve between $95.24$ and $97.62$ percent accuracy for the inside case and between $45.33$ and $78.67$ percent accuracy for the outside scenario. 
As mentioned in \ref{sec:ChoiceOfTechnique}, the goal of the project was not to achieve high accuracy and precision on the exact positioning of people in a room. 
The goal was rather to investigate whether we could detect the presence of people in a meeting with reasonable certainty.

These findings align with what similar research has concluded; that using kNN is sufficiently accurate, and that complex machine learning is not necessary to obtain useful results.\cite{ble_kneares_neural}

The experimental setup depicted in Figure~\ref{fig:experiment_setup} places four beacons within a room.
Despite the constraints of time and budget, the setup yielded valuable data and insights.

Looking towards future investigations, the placement of beacons outside the room would be worth investigating.
Such a configuration could potentially impact the predictive accuracy of the model.
Similarly, varying the total number of beacons, both inside and outside the room, could impact data richness and consequently, the precision of predictions.

Furthermore, extending the grid points within and beyond the room could also yield benefits. By adjusting the granularity of data collected, we may observe a corresponding change in the precision of the predictive model. 

However, it is worth noticing that for measuring points \textit{B} and \textit{J} in figure \ref{fig:experiment_room}, a large uncertainty in classification is observed.
We suspect that this is a consequence of the glass pane next to the door having a much lower absorbation of radio frequencies than the door and the wall next to it.
This low absorbation results in the two data points next to the pane having similar distances to both map values labelled as \textit{inside} and some as \textit{outside}, therefore giving uncertain classifications.

A limitation with the presented approach is that it requires employees to carry a device capable of running the app in the background.
This makes the approach unsuitable for organizations that do not provide their employees with such devices.
It is a tradeoff between convenience and accuracy.
Proximity detection, for example, would have been more convinient, since it would only require employees to carry a Bluetooth-enabled devices (such as a mobile phone or Bluetooth tag), however this method is far less accurate, as discussed in Section~\ref{sec:proximity_detection}.


In conclusion, while the current setup has provided significant insights, these proposed modifications - additional beacons and expanded grid points both within and outside the room - suggest promising avenues for future research to further refine the predictive accuracy of our model.
We believe that our findings provide sufficient evidence that radio mapping with a classification algorithm like kNN can be used to reliably detect the presence of an individual in a meeting room and thus indicate the meeting status.