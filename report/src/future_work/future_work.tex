\chapter{Future Work}
As mentioned in Section \ref{sec:fingerprinting}, the number of beacons and their placement affects the quality of the localization results when using fingerprinting. 
Similarly, the physical layout and the obstacles within a room can affect the classification performed in the second phase of fingerprinting. 
It is therefore relevant to experiment with several configurations related to beacon placement, number of beacons, physical room layout, and different radio wave noise within the room and in the surrounding environment. 

Future research efforts in room occupancy detection using fingerprinting via BLE should focus on finding a setup that demonstrates improved adaptability in a variety of environments.
One approach to achieving this involves further studies of strategies that can improve either the map, or the classification algorithm.
Our experiments showed that the quality of classifications degraded for positions outside the room by up to $49.91$ percent compared to indoor positions, as outlined in Table \ref{lst:resultsPhone_precision}.
Therefore, future research should experiment with different beacon configurations where, for example, one or more beacons are placed outside the meeting room. 
Future experiments might show an improvement in the map quality, resulting in better classification. Alternatively, it could show that fewer beacons are needed to achieve similar results.
Building on this, exploring alternate algorithms could be beneficial for advancing the adaptability of our system.
Potential candidates might include statistical kNN methods or machine learning approaches like support vector machines, similar to what \citeauthor{ble_kneares_neural} proposed in their paper \cite{ble_kneares_neural}. 
By gaining more insight into the impacts of algorithm and configuration variations in the experimental setup one might be able to achieve results that can be used to create a more generalized solution.

In our proposed system for Mariendal IT, the results found during experimentation process could be incorporated into a system similar to the prototype detailed in Section \ref{subsec:develop_proto},
particularly in relation to the optimal algorithm.

In our prototype evaluation, we identified the need for Mariendal IT's meeting room management system to be capable of tracking meeting attendees, as evidenced in Appendix \ref{appendix:user_stories} and \ref{appendix:usability_tasks}. 
To address this, a mobile application could be developed for use on participants' phones.
A possible function of this application could be to have it work primarily as a receiver of beacon signals.
The application could then relay its latest received measurements to a server. 
This would enable the server, using pre-existing datasets specific to the individual meeting rooms, to compute and predict the presence of users within the meeting rooms. 
By establishing such a client-server relationship between the user's phones and a server through the application, and delegating the classification to the server, the tablet outside the meeting room would be able to more accurately assess whether the meeting room is currently occupied or not.
Furthermore, the system would then be enabled to automatically end a meeting when attendees leave the room, potentially improving the management of meeting rooms.

