\section{Sprint length and definition of done}
To accommodate the agile development method three things must be established.
First, the duration of sprints must be decided.
Second, a definition of done must be established by the Scrum team.
After this is done the team will decide on who will be product owner and scrum master during the project. 
The duration of the sprints must be chosen such that they are long enough for the Scrum team to provide a finished product, but short enough to allow the team to adapt to changes\cite{scrumguide}. 
We have chosen a sprint length of three weeks.
This will allow us enough time to develop useful functionality of meaningful sizes but does not limit us in terms of adaptability.

The team's will include two definitions of done - one for deciding when a user story is satisfied and one for deciding when subtasks of the users stories are done.
\begin{dod}
    For a user story to be complete two requirements must be satisfied: An end-to-end test must carried out to ensure that the functionality needed to satisfy the user story is completed. Secondly, the implementation of the user story must be documented. 
\end{dod}
To ensure that the definition of done for user stories are satisfied each story has an owner. 
The owner must ensure that the definition of done is satisfied before the future is deemed complete.

For minor tasks the following the definition of definition of done is used. 
\begin{dod}
    All written code must be annotated with types and a linter used to ensure the quality of the code.
    Before any code is merged into the existing code base using version control, it must be tested that the system compiles and build successfully.  
\end{dod}

