\section{Implementing Agile Project Management with Scrum}
Developing a software product for a business can be a complex process.
Incremental agile approaches have become popular to address this complexity, as they allow for faster development and adaptation to changes.\cite{sommervilleSoftwareEngineering2016}
In incremental agile approaches, a system is developed as a series of increments, each adding functionality to the system.
Between these increments, one gathers feedback from users to prioritize functionality for the next increment.
Therefore, each increment adds value to the user.
By continuously getting feedback and working with these increments, the development process becomes cheaper and faster as a result of less time spent reworking the system.\cite{sommervilleSoftwareEngineering2016}
This section will discuss how we are using an agile approach for our project, focusing specifically on the Scrum framework.

\subsection*{Scrum Framework}
Agile is a set of values, and does not present practical activities which a process model would consist of\cite{sutherlandScrumArtDoing2014}.
Therefore, the team has decided on Scrum\cite{scrumdotorg}, which is an agile framework that puts the values to practice.
This decision was made due to prior experience with using the framework amongst team members.

Scrum introduces a variety of new practices and roles, which we will describe in the following sections.

\subsubsection*{Roles and Responsibilities}
In a Scrum team, there is a Scrum Master, a Product Owner, and developers.
The Scrum Master is responsible for enforcing scrum and helping the team remain productive.
Their goal is to guide the team towards continuous improvement.
A Product Owner is responsible for maximizing value from the product.
They are also accountable for managing the product backlog, which includes developing a product goal, creating backlog items and prioritizing them, and making sure the team understands this backlog.

\subsubsection*{Scrum Artifacts and Tools}
We will use various artifacts and tools during the Scrum process, including the product backlog, user stories, sprints, sprint backlog, sprint goal, increments, definition of done, daily scrums, sprint review, and sprint retrospective.
These elements help ensure the project is organized, focused, and continuously improving.

The \textbf{product backlog} is a list of user stories, and represents every possible addition to the project.
Due to the comprehensive nature of this list, it is not possible to build everything.
Therefore, the list should be prioritized by what delivers the most value to users.
Backlog items then are estimated by the team in terms of relative size.
A common measure for estimation is the Fibonacci sequence.
As the stakeholders provide feedback, items in the backlog should be prioritized accordingly.
All these items contribute to achieving the \textbf{product goal}.
The backlog items are \textbf{user stories}.
These stories represent functionality the stakeholder would like to see in the system.
For this reason, they must include \textit{who} it is for, \textit{what} should be done, and \textit{why} they want it.
This is captured in the general "\texttt{As a X, I want Y, such that Z.}" format for user stories.
A \textbf{sprint} is a short period of time, during which the team works on a subset of the product backlog, captured in a \textbf{sprint backlog}.
This defines the \textit{what} of the sprint.
The items in the sprint backlog are chosen based on a \textbf{sprint goal}, defining the \textit{why} of the sprint.
The sprint also has an actionable plan for delivering the increment, representing the \textit{how}.
At the beginning of teach sprint, the Scrum team performs \textbf{sprint planning}.
Here, the team discuss and choose the most important backlog items for the sprint.
Then they find a \textbf{sprint goal} which clarifies why the sprint is valuable to the users.
The team also discussed how the chosen work will get done in increments that meet the definition of done.
\textbf{Increments} can be viewed as stepping stones toward the product goal.
Each results in a product that is closer to reaching the product goal.
A \textbf{definition of done} describes the minimum requirements that must be fulfilled before an increment can be considered completed.
Usually, simply implementing a feature is not sufficient to meet this definition.
Testing, ensuring deployability, and writing documentation may also be required.
As the sprint progresses, the team hold \textbf{daily scrums}, which are short meetings where the team inspects progress towards the sprint goal and adjust the sprint backlog as necessary.
Daily scrums are meetings where progress towards the sprint goal is discussed.
Each increment will be inspected during \textbf{sprint review}.
During such a review, progress is examined, the team and its stakeholders discuss any changes, and then figure out what to do next.
Finally, the team conducts a \textbf{sprint retrospective}, where the team examines how their last sprint went, and how they may improve.
This is part of the quest for continuous improvement.\cite{sutherlandScrumArtDoing2014}

\subsubsection*{Scrum Process}
In short, the process for working as a Scrum team involves repeatedly taking backlog items and working on them during a sprint.
After each sprint, the team and stakeholders inspect the results and adjust for the next sprint.
This fosters continuous improvement by getting rapid feedback on work.

Our team has chosen to work in three-week sprints.
This will allow us enough time to develop useful functionality of meaningful sizes but does not limit us in terms of adaptability.

The team's will include two definitions of done - one for deciding when a user story is satisfied and one for deciding when subtasks of the users stories are done.
\begin{dod}
    For a user story to be complete two requirements must be satisfied: An end-to-end test must carried out to ensure that the functionality needed to satisfy the user story is completed. Secondly, the implementation of the user story must be documented. 
\end{dod}
To ensure that the definition of done for user stories are satisfied each story has an owner. 
The owner must ensure that the definition of done is satisfied before the future is deemed complete.

For minor tasks the following the definition of definition of done is used. 
\begin{dod}
    All written code must be annotated with types and a linter used to ensure the quality of the code.
    Before any code is merged into the existing code base using version control, it must be tested that the system compiles and build successfully.  
\end{dod}
