This research endeavor was undertaken in collaboration with Mariendal IT, an organization in the information technology domain that specializes in the provision of hosting, security, and operations management services\cite{Mariendal_OmOs}.
As a key stakeholder, Mariendal IT contributed to the identification of their unique requirements, offered feedback, and partook in prototype- and user interface evaluations through consultations.

This research addresses Mariendal IT's meeting room management problem by developing a presence detection system to automatically detect people's presence in meeting rooms.
The purpose of the system is to provide Mariendal IT with a means of determining the occupancy status of meeting rooms, thereby enabling them to implement a meeting room management solution that can be used to optimize the utilization of meeting rooms.
This way, Mariendal IT can ensure that meeting rooms are being used efficiently, reducing the costs associated with maintaining underutilized meeting rooms.

Presence detection systems utilize a diverse array of technologies and strategies in order to determine the location of individuals or devices within a given vicinity.
The selection of an appropriate technology is contingent upon a variety of factors, including the accuracy and precision of location determination, susceptibility to environmental factors, privacy considerations, and implementation costs.
After evaluating these criteria, Bluetooth Low Energy (BLE) emerged as the technology of choice for this project.

This report presents an analysis of various methods and techniques for presence detection using BLE technology.
A suitable method was then identified for implementation in this project, with fingerprinting utilizing a k-nearest neighbor (kNN) approach deemed to be the most appropriate choice.

The final phase of the research involved the evaluation of the presence detection system's efficacy in determining the occupancy status of meeting rooms.
The outcomes of this assessment are presented and discussed, providing insights into the system's overall performance and potential for further enhancement.
