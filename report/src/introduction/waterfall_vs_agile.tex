\section{Implementing Agile Project Management with Scrum}
Developing a software product for a business can be a complex process.
Incremental agile approaches have become popular to address this complexity, as they allow for faster development and adaptation to changes.\cite{sommervilleSoftwareEngineering2016}
In incremental agile approaches, a system is developed as a series of increments, each adding functionality to the system. 	
Between these increments, one gathers feedback from users to prioritize functionality for the next increment.
Therefore, each increment adds value to the user.
By continuously getting feedback and working with these increments, the development process becomes cheaper and faster as a result of less time spent reworking the system.\cite{sommervilleSoftwareEngineering2016}
This section will discuss how we are using an agile approach for our project, focusing specifically on the Scrum framework.

\subsection*{Scrum Framework}
Agile is a set of values, and does not present practical activities which a process model would consist of\cite{sutherlandScrumArtDoing2014}.
Therefore, we have chosen Scrum\cite{scrumdotorg}, an agile framework that puts these values into practice.
This decision was made due to prior experience with using the framework amongst team members.

\subsubsection*{Roles and Responsibilities}
Scrum introduces a variety of new practices and roles.
In a Scrum team, there is a Scrum Master, a Product Owner, and developers.
The Scrum Master is responsible for enforcing scrum and helping the team remain productive.
A Product Owner is responsible for maximizing value from the product and managing the product backlog.

\subsubsection*{Scrum Process}
The process for working as a Scrum team involves repeatedly taking backlog items and working on them during a sprint.
After the sprint, the team and stakeholders inspect the results and adjust for the next sprint.
This fosters continuous improvement by getting rapid feedback on work.

\subsubsection*{Scrum Artifacts and Tools}
We will use various artifacts and tools during the Scrum process, including the product backlog, user stories, sprints, sprint backlog, sprint goal, increments, definition of done, daily scrums, sprint review, and sprint retrospective.
These elements help ensure the project is organized, prioritized, and continuously improving.

\subsection*{Gathering Initial Requirements}
In the next section, we will describe how we gathered initial requirements for the project and created user stories for the product backlog.

