\section{Presence detection using $\text{CO}_{2}$ }\label{sec:presence_env}
$\text{CO}_{2}$ falls under the category of an \textit{environmental system}. 
These system measure physical attributes of an environment, and reacts according to these measurements.
Therefore, such systems can be used to infer occupancy of rooms by detecting changes in environmental parameters caused by human presence.
$\text{CO}_{2}$ sensors does this by measuring the level of $\text{CO}_{2}$ in the atmosphere within the room \cite{longoAccurateOccupancyEstimation2019, gruberCO2SensorsOccupancy2014}.

\citeauthor{longoAccurateOccupancyEstimation2019}\cite{longoAccurateOccupancyEstimation2019} summarizes a number of methods for occupancy detection using $\text{CO}_{2}$. 
The results shows accuracy of $\text{CO}_{2}$ measurements for occupancy detection is between 50\% and 99\%, depending on task and amount of occupants to estimate.
%Is it easier to detect many occupants?

For $\text{CO}_{2}$ and other environmental systems, the accuracy of the prediction of whether a room is occupied or not, is vulnerable to even small changes to the physical environment in which the system resides.\cite{gruberCO2SensorsOccupancy2014,longoAccurateOccupancyEstimation2019}
Particularly, the ventilation flow rate of the room affects its $\text{CO}_{2}$ concentration. Opening doors or windows will thus have an effect on the accuracy of the approach.
Since changes in $\text{CO}_{2}$ levels of the room are not immediatedly noticable, a system using this to detect occupancy will not be quickly detect a change in occupation.
This can be handled by analysing the gradient of increase in $\text{CO}_{2}$ levels\cite{gradient_co2}.
However, this gradient change is dependent on many variables, such as room size or number of people in the room. 
