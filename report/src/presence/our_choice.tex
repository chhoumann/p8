\section{Choice of technology}
In this Section, we will compare the technologies presented in Sections \ref{sec:presence_env} to \ref{sec:bluetooth}, and decide which technology best satisfies the criterias listed in Table \ref{lst:presence_eval_criteria}. 

Section \ref{sec:presence_env} presents $\text{CO}_{2}$ sensoring as an occupation detection technology. The technology can accurately detect the presence in a room using by measuring $\text{CO}_{2}$ levels.
The technology suffers from unpredictability, due to its dependency on the environmental context of the system.
For instance, $\text{CO}_{2}$ levels change depending on whether or not windows are open, how many participants are attending the meeting, and how long the meeting is.
Thus, the accuracy of the technology depends on external factors that the technology cannot control.
Depending on the external factors, changes to the environmental indicator used to detect presence might not be promptly detectable.
As this technology focuses solely on the measurements of environmental factors, the privacy of the users of the system is protected, as information about the users is not collected.

Section \ref{sec:video-based-detection} presents video-based technologies.
As most of the technologies are based on a live video feed, one can accurately determine the position of occupants within the room with high level of precision. 
However, this reliance on live video leads to privacy concerns.
Many of these technologies requires that the environment in which they are deployed to be suitable.
If the shape of the room is not rectangular, multiple cameras might be necessary to fully cover angles of the room, to ensure that occupants do not go unnoticed. 
Similarly, the meeting room can be inadequately lit, decreasing the accuracy of the detection.
If this not the case, infrared technology can be utilized to counteract the inadequate environment, but this increases the cost of the technology.   

Section \ref{sec:wi-fi} and Section \ref{sec:bluetooth} presents two radio technologies. 
WiFi can be used to identify individual devices that might be stationary or mobile by having them broadcast a signal containing the identity of the device, and track the strength of this signal.
To ensure devices broadcast a signal with a non-changing identity, management and re-configuration of the devices is required.
The identity of these devices must be also stored responsibly to not raise privacy concerns.
If multiple devices are carried by the same person, it can be wrongfully interpreted as multiple persons close to each other. 
However, as WiFi is widely used in office environments, its availability is high and the cost of installment and management negligible.
Bluetooth exists in two variants: Bluetooth Classic and Bluetooth Low Energy.
Bluetooth Classic is sensitive to obstructions which raises concerns about reliability and accuracy due to environmental circumstances. 
For instance, objects placed in the meeting room could disrupt the detection of devices and therefore make the technology unreliable.
BLE, on the contrary, has fewer limitations concerning obstructions and is therefore less dependent on the physical configuration of the external environment. 
Therefore, it has higher reliability.
Multiple devices using either BLE or Bluetooth Classic can be carried by the same person, which may cause issues with accuracy.
However, existing hardware for presence detection, such as BLE beacons, could be an affordable approach to increasing the accuracy of the technology.  
Similar to WiFi, collected device information should be stored responsibly.
Personal health devices can broadcast messages containing sensitive information.

\todo{fix table}
Table \ref{lst:evaluated_technologies} summarizes the characteristics of the covered technologies in terms of the chosen metrics seen in Table \ref{lst:presence_eval_criteria}.
\begin{table}[H]
    \centering
    \begin{center}
        \begin{tabular}{|l|p{20mm}|p{20mm}|l|l|p{25mm}|}
            \hline
            Technology/Metrics & Accuracy & Precision & Cost & Privacy & Environmental robustness \\ \hline
            Bluetooth          & x        & x         & x    & x       & -                      \\ \hline
            Wifi               & x        & x         & -    & x       & -                      \\ \hline
            CO2 sensors        & Not applicable        & Not applicable         & x    & x       & -                      \\ \hline
            Camera (regular)   & x        & x         & x    & -       & -                      \\ \hline
            Camera (infrared)  & x        & x         & -    & x       & x                      \\ \hline
        \end{tabular}
    \end{center}
    \caption{Summerized capabilities of the technologies using the metrics from Table \ref{lst:presence_eval_criteria}. X marks capabilities satisfied by the technology}
    \label{lst:evaluated_technologies}
\end{table}






As $\text{CO}_{2}$ sensors are dependent on environmental indicators that cannot be controlled, the accuracy of a solution utilizing this technology might not be sufficient.
Even though video technologies provide a high level of accuracy and are mostly independent of environmental factors, such as air circulation radio wave fluctuations, the privacy concerns, cost of installment, and maintenance are deemed too high.
WiFi is already an integral part of any office environment. 
A system analysing broadcasted identity of devices is not acceptable, as the cost of reconfiguring personal devices is too high and might not be possible.  
BLE supports many of the same use cases as WiFi, but does not rely on the devices broadcasting their identity regularly, but instead, require the device to collect information broadcasted by the system.
Additionally, existing technologies supporting accurate presence detection using BLE exist, making it cheap to configure and manage. 
Therefore, BLE is a suitable technology to create a system to accurately examine whether a meeting room is being occupied.
In the following chapter, we examine in detail different approaches to presence detection using BLE.