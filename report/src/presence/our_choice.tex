\section{Choice of technology}
In this Section, we will compare the technologies presented in Sections \ref{sec:presence_env} to \ref{sec:bluetooth}, and decide which technology best satisfies the criterias listed in Section \ref{lst:presence_eval_criteria}. \todo{The listing does not have a ref. I dont know what "Listing 3" is.}

In section \ref{sec:presence_env}, Environmental-based technologies was described. These technologies can accurately detect the presence in a room using multiple environmental sources, such as $\text{CO}_{2}$ level, light, humidity and temperature.
A drawback of environmental-based technologies is that they suffer from unpredictability.
For instance, $\text{CO}_{2}$ levels change depending on whether or not windows are open.
Thus, the accuracy of the technology can depend on external factors that the technology cannot control.
Depending on the external factors, changes to the environmental indicator used to detect presence might not be promptly detectable.
As this technology focuses solely on the measurements of environmental factors, the privacy of the users of the system is protected, as information about the users is not collected.

Section \ref{sec:video-based-detection} presents video-based technologies.
As most of the technologies are based on a live video feed, one can precisely determine whether a room is occupied responsively. 
Consequently, this reliance on live video leads to privacy concerns.
Many of these technologies are fully reliant on rooms being suitably lit.
However, this can be counteracted using infrared technology in the camera, but this increases the cost of the technology.   

Section \ref{sec:wi-fi} and Section \ref{sec:bluetooth} presents two radio technologies. 
WiFi can be used to identify individual devices that might be stationary or mobile by having them broadcast a signal containing the identity of the device, and track the strength of this signal.
To ensure devices broadcast a signal with a non-changing identity, management and re-configuration of the devices is required.
The identity of these devices must be also stored responsibly to not raise privacy concerns.
If multiple devices are carried by the same person, it can be wrongfully interpreted as multiple persons close to each other. 
However, as WiFi is widely used in office environments, its availability is high and the cost of installment and management negligible.
Bluetooth exists in two variants: Bluetooth Classic and Bluetooth Low Energy.
Bluetooth Classic is sensitive to obstructions which raises concerns about reliability and accuracy due to environmental circumstances. 
For instance, objects placed in the meeting room could disrupt the detection of devices and therefore make the technology unreliable.
BLE, on the contrary, has fewer limitations concerning obstructions and is therefore less dependent on the physical configuration of the external environment. 
Therefore, it has higher reliability.
Multiple devices using either BLE or Bluetooth Classic can be carried by the same person, which may cause issues with accuracy.
However, existing hardware for presence detection, such as BLE beacons, could be an affordable approach to increasing the accuracy of the technology.  
Similar to WiFi, collected device information should be stored responsibly.
Personal health devices can broadcast messages containing sensitive information.

\todo{fix table}
Table \ref{lst:evaluated_technologies} summarizes the characteristic of the covered technologies in terms of the chosen metrics seen in Table \ref{lst:presence_eval_criteria}.
\begin{table}[H]
    \centering
    \begin{center}
        \begin{tabular}{|l|p{20mm}|p{20mm}|l|l|p{25mm}|}
            \hline
            Technology/Metrics & Accuracy & Precision & Cost & Privacy & Environmental robustness \\ \hline
            Bluetooth          & x        & x         & x    & x       & -                      \\ \hline
            Wifi               & x        & x         & -    & x       & -                      \\ \hline
            CO2 sensors        & Not applicable        & Not applicable         & x    & x       & -                      \\ \hline
            Motion detection   & Not applicable        & Not applicable         & x    & x       & x                      \\ \hline
            Camera (regular)   & x        & x         & x    & -       & -                      \\ \hline
            Camera (infrared)  & x        & x         & -    & x       & x                      \\ \hline
        \end{tabular}
    \end{center}
    \caption{Summerized capabilities of the technologies using the metrics from Table \ref{lst:presence_eval_criteria}. X marks capabilities satisfied by the technology}
    \label{lst:evaluated_technologies}
\end{table}






As environmental sensors are dependent on environmental indicators that cannot be controlled, the accuracy of a solution utilizing this technology might not be sufficient.
Even though video technologies provide a high level of accuracy and are mostly independent of environmental factors, the privacy concerns and cost of installment and maintenance are deemed too high.
WiFi is already an integral part of any office environment, but relying on the reconfiguration of personal devices is not accept    cable. 
BLE supports many of the same use cases as WiFi, but does not rely on the devices broadcasting their identity regularly.
Additionally, existing technologies supporting accurate presence detection using BLE exist, making it cheap to configure and manage. 
Therefore, BLE is a suitable technology to create a system to accurately examine whether a meeting room is being used.

In the following chapter, we examine in detail different approaches to presence detection using BLE.