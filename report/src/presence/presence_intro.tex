\chapter{Room Occupancy Detection Technologies}\label{chap:presence_intro}
To address the problem defined in Section \ref{sec:problem_statement}, we propose an occupancy detection system (presence system) to identify when a meeting starts, defined as participants entering a room, and when it ends, defined as all participants leaving the meeting room.
This chapter examines various technologies that can be used in occupancy detection, evaluates them, and identifies which one of them could be used in the suggested system.
The chosen technology will serve as the basis for the rest of the system, which will be described in the following sections of the report.

Several technologies can be utilized for detecting room occupancy, for instance, infrared (IR) sensors\cite{woodward-2021, dodierBuildingOccupancyDetection2006, OccupancySensorMotion}, ultrasonic sensors\cite{woodward-2021, dodierBuildingOccupancyDetection2006, OccupancySensorMotion}, microwave sensors\cite{woodward-2021}, computer vision\cite{co2sensor, longoAccurateOccupancyEstimation2019, OccupancySensorMotion}, Bluetooth and Wi-Fi\cite{teissedre-2019}, pressure sensors\cite{OccupancySensorMotion}, acoustic sensors\cite{OccupancySensorMotion}, and CO2 sensors\cite{co2sensor, longoAccurateOccupancyEstimation2019, jinSensingProxyOccupancy2015}.\cite{faragherLocationFingerprintingBluetooth2015}
The following sections will examine a few of them-based.

\citeauthor{presence_ble_review} \cite{presence_ble_review} dictates a number of performance metrics for presence based systems. Of these, we adapt accuracy, precision, and cost, but also include considerations for responsiveness, privacy, and environmental factors when evaluating the technologies.
because they cover what we believe are the key aspects necessary for the successful implementation of a system such as this.
Accuracy and responsiveness ensure reliable detection and efficient space management, while privacy addresses user confidence and regulatory compliance.
Cost ensures affordability and sustainability, and environmental factors guarantee consistent performance under varying conditions.
Taken together, these criteria help to select a technology that fits the context of use.
A short definition of each critera is given in Table~\ref{lst:presence_eval_criteria}.



\begin{table}[H]
    \begin{tabular}{|l|p{85mm}|lll}
    \cline{1-2}
    \textbf{Metrics} & \textbf{Description} &  &  &  \\ \cline{1-2}
    Accuracy         & The ability to correctly detect the location of a person or object &  &  &  \\ \cline{1-2}
    Precision        & The ability to consistently estimate the predicted location                           &  &  &  \\ \cline{1-2}
    Cost             & The expenses associated with acquiring, installing, and maintaining the technology &  &  &  \\ \cline{1-2}
    Privacy          & Ensuring that the detection method respects the privacy of meeting participants    &  &  &  \\ \cline{1-2}
    Environmental Robustness    & The technology's robustness to various environmental conditions affecting its accuracy or precision, for instance light level, obstructions, or configuration possibilities for the technology.                     &  &  &  \\ \cline{1-2}
    \end{tabular}
    \caption{Presence evaluation criteria}
    \label{lst:presence_eval_criteria}
\end{table}

In this project, we work with the classification of whether a meeting room is occupied or not. 
A system for this purpose is suitably accurate, if the determined location can be used to correctly classify if the room can be marked as unoccupied.
It is sufficient to mark the room as unoccupied when the system is certain that all attendees is not in the room. 
In this regard, it is not necessary for the system to detect the exact moment an attendee leaves the meeting room.
The system can wait until a satisfiable distance to the room is reached, before marking it as occupied. 
This gives leverage in terms of the accuracy for the solution. 
In this project, an accuracy of x \todo{definer hvad en suitable accuracy er} meters has been determined based on stakeholder information.  