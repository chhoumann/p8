\section{Environmental systems}\label{sec:presence_env}
Environmental systems for occupancy detection leverage a variety of sensors that measure physical attributes of an environment, such as carbon dioxide ($\text{CO}_{2}$) concentration, temperature, humidity, and infrared radiation.
These systems infer occupancy by detecting changes in environmental parameters caused by human presence.
For instance, an increase in $\text{CO}_{2}$ levels may indicate occupancy due to human respiration.\cite{longoAccurateOccupancyEstimation2019}
\subsection{Presence detection using $\text{CO}_{2}$ concentration}\label{sec:presence_env_co2}
As illustrated in the example in section~\ref{sec:presence_env}, it is possible to use $\text{CO}_{2}$ levels to detect presence in a room.
This approach uses $\text{CO}_{2}$ sensors to detect the presence of a person in a room, and is based on the assumption that the presence of a person will increase the $\text{CO}_{2}$ concentration in the room.\cite{gruberCO2SensorsOccupancy2014}

However, there are several causes of concern with this approach. Some notable ones are:\cite{gruberCO2SensorsOccupancy2014,longoAccurateOccupancyEstimation2019}
\begin{itemize}
    \item Ventilation flow rate matters as it affects the $\text{CO}_{2}$ concentration in the room. However, in some areas, this could be outside of the control of the user.
    \item Similarly, opening a door or window will also affect the $\text{CO}_{2}$ concentration in the room.
    \item There is a time delay between the presence of a person and a noticeable increase in $\text{CO}_{2}$ concentration. The more people that are present, the shorter the time delay will be.
\end{itemize}

Therefore, the environmental prerequisite for this approach is that the ventilation flow rate is rather constant, and the room is not exposed to a lot of outside air during measurements.
This may be difficult to achieve in some environments, such as offices, where the ventilation flow rate is often not constant, and the room is exposed to a lot of outside air, given open windows or doors.

Various studies have been conducted to determine the accuracy of $\text{CO}_{2}$ sensors for occupancy detection. 
\citeauthor{longoAccurateOccupancyEstimation2019}\cite{longoAccurateOccupancyEstimation2019} summarizes some of these and show the accuracy of $\text{CO}_{2}$ measurements for occupancy detection is between 50\% and 99\%, depending on task and amount of occupants to estimate.

\section{Video-based systems}\label{sec:presence_video}
Video-based systems for occupancy detection utilize various imaging technologies and techniques to identify the presence of people within a specified area. 
These system uses the adaptability of cameras to capture visual data, that can be used detect motion using approaches like machine learning. 
These approaches allow video-based systems to accurately estimate the number of people in a monitored space.
In the following section, we examine the use of video-based technologies for presence detction, discussing the advantages and disadvanteges with various methods.
\subsection{Video-based detection} \label{sec:video-based-detection}
Cameras are a common technology used to detect the presence of humans in a room.
There are many different approaches to how one might use a camera for presence detection.

One of the primary methods for presence detection is the use of machine learning techniques.
These techniques work by training a machine learning model, such as a Convolutional Neural Network (CNN).
The resulting model can then be used to analyze images and detect objects in them.
For example, \citeauthor{FUERTES2022103473}\cite{FUERTES2022103473} used an omnidirectional camera to gather images of a room, and then trained a CNN to detect the presence of people in these images.
The training is done through point-based annotation of the images as opposed to the more common approach of using bounding box annotation.
Essentially, images are annotated with points denoting the location of the person in the image as opposed to drawing a bounding box around the person.
This approach is advantageous because it is faster to annotate images in this way, but it is also far more accurate at representing the position of a human on images taken with an omnidirectional camera, which are often distorted or skewed.\cite{FUERTES2022103473}

Motion detection is another commonly used to detect the presence of people in a room, and is defined by \citeauthor{ANANDAN1988347}\cite{ANANDAN1988347} as "\textit{the task of detecting movement of image elements across the image plane}".
In other words, motion detection is a broad term that can refer to a variety of different methods.
For example, motion detection can be achieved by comparing two images of the same scene, and detecting the differences between them\cite{granath_detecting_nodate}.

Alternatively, one may also utilize passive infrared (PIR) sensors to detect motion.
PIR sensors are able to detect motion by measuring the infrared radiation emitted by objects in the room.\cite{Deiana2014}

Many other methods exist, and the choice of method will ultimately depend on the specific application and system requirements. 
Regardless of the method, a number of key considerations apply when using cameras for presence detection, including the sensitivity of the devices to lighting, obstructions, movement, temperature, and detection range and area.
For example, if the device is obstructed by some object, it may be unable to gather useful data.
For cameras specifically, consider how it may not be able to get a clear image of a room if the lighting is too dark or too bright.
Most current systems also use standard perspective cameras with a narrow field of view that can only detect people in a limited area in front of the camera\cite{FUERTES2022103473}.
This, combined with positioning and orientation of the camera, can be detrimental to the quality of the data.\cite{granath_detecting_nodate, tang_occupancy_2020}

These limitations can increase the complexity of the system, as one would have to take these factors into account during system design and development, and when deciding which devices to use as well as where to place them.
Some limitations can be overcome by using multiple sensors or cameras, but this futher increases the cost.\cite{FUERTES2022103473}

Using cameras for presence detection also raises privacy concerns.
Being under surveillance by a camera is problematic, as it can be seen as a violation of privacy.
Consequently, applications employing cameras for presence detection must include a privacy policy that outlines how the data is collected, stored, and used, which all employees must also agree to.
These requirements add another layer of complexity to the system, as the privacy policy must be implemented and enforced by the system and the system owner.
Moreover, because of these privacy concerns, acceptance and subsequent use of the system may be limited.\cite{granath_detecting_nodate, tang_occupancy_2020, PrivacyPreservingSensor}

\section{Radio-based systems}\label{sec:presence_radio}
Radio-based occupancy detection systems make use of wireless communication technologies, such as Wi-Fi and Bluetooth, to determine the presence of individuals in a given area.
These systems exploit the fact that most people carry devices that emit wireless signals like smartphones and wearables.

By monitoring the strength and distribution of these signals, radio-based systems can infer the number of occupants and their locations within the monitored area.\cite{longoAccurateOccupancyEstimation2019, teissedre-2019}
