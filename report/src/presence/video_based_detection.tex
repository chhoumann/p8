\subsection{Object recognizion} \label{sec:video-based-detection}\
Methods for object recognition rely on feeding a system with one or more images \cite{huttenlocher1987object, computational_strategies_for_object_recog}, and analysing these images to detect whether an object is present on the image.
Several technologies can be utilized to obtain these images, but one of the most common is camera technologies\cite{FUERTES2022103473}. 
Approaches to object recognition often utilize machine learning to create models that establish whether or not some image contains a specific object\cite{huttenlocher1987object}, and can utilize several computational strategies to obtain the results\cite{computational_strategies_for_object_recog}. 
Several approaches using machine learning can be utilized. 
\citeauthor{object_recognision_survey} \cite{object_recognision_survey} describes several methods, their accuracy, and under which conditions the methods to work well. 
However, the quality of the image or video data in object recognition can effect the results drastically.
\citeauthor{vid_qual_affects} \cite{vid_qual_affects} details how approaches using deep neural networks cannot reliably recognize objects if trained on data of higher quality. 
Thus, systems using this approach rely on using cameras that supply the system with images of similar quality to that which the system was trained on.
This also means that the approach is substandard when deployed in an environment with too much or too little light, as this can affect image quality.  
The nature of using cameras also raises some concerns with privacy, due to the approach relying on video information. 
Consequently, applications employing cameras for presence detection must include a privacy policy that outlines how the data is collected, stored, and used, which all meeting attendants must agree to.
These requirements add another layer of complexity to the system, as the privacy policy must be implemented and enforced by the system and the system owner.
The system must also keep up with changes to the regulations dictating these policies.
Because of these privacy concerns, acceptance and subsequent use of the system may be limited.\cite{granath_detecting_nodate, tang_occupancy_2020, privacyPreservingSensor}.
Anonymity could be partially preserved by using infrared cameras, as the video feed from such cameras does not supply the same degree of details that can be used to identify individuals. 
This would also eliminate the need for adequate lighting where the system is deployed.  
However, using infrared cameras will raise the price of the system significantly. 

\subsection{Motion detection}
Another image-based approach to detecting whether a room is being used is motion detection.
\citeauthor{ANANDAN1988347}\cite{ANANDAN1988347} defines motion detectoin as "\textit{the task of detecting movement of image elements across the image plane}".
Thus, motion detection is a broad term that can refer to a variety of different methods for detecting any object's movement. 
For example, motion detection can be achieved by comparing two images of the same scene, and detecting the differences between them\cite{granath_detecting_nodate}.

Alternatively, one may also utilize passive infrared (PIR) sensors to detect motion.
PIR sensors are able to detect motion by measuring the infrared radiation emitted by objects in the room.\cite{Deiana2014}

Many other methods exist, and the choice of method will ultimately depend on the specific application and system requirements. 
Regardless of the method, a number of key considerations apply including the sensitivity of the devices to lighting, obstructions, movement, temperature, and detection range and area.
For example, if the device is obstructed by some object, it may be unable to gather useful data.
For cameras specifically, consider how it may not be able to get a clear image of a room if the lighting is too dark or too bright.
Most current systems also use standard perspective cameras with a narrow field of view that can only detect people in a limited area in front of the camera\cite{FUERTES2022103473}.
This, combined with positioning and orientation of the camera, can be detrimental to the quality of the data.\cite{granath_detecting_nodate, tang_occupancy_2020}

These limitations can increase the complexity of the system, as one would have to take these factors into account during system design and development, and when deciding which devices to use as well as where to place them.
Some limitations can be overcome by using multiple sensors or cameras, but this futher increases the cost.\cite{FUERTES2022103473}

Using cameras for presence detection also raises privacy concerns.
Being under surveillance by a camera is problematic, as it can be seen as a violation of privacy.
Consequently, applications employing cameras for presence detection must include a privacy policy that outlines how the data is collected, stored, and used, which all employees must also agree to.
These requirements add another layer of complexity to the system, as the privacy policy must be implemented and enforced by the system and the system owner.
Moreover, because of these privacy concerns, acceptance and subsequent use of the system may be limited.\cite{granath_detecting_nodate, tang_occupancy_2020, PrivacyPreservingSensor}