\subsection{Wi-Fi} \label{sec:wi-fi}
Wi-Fi is a wireless networking technology allowing devices to communicate with each other and access the Internet without the need for cable connections.
It is based on the IEEE 802.11 standards and uses radio waves to transmit data between devices.
Wi-Fi allows devices such as computers, smartphones, printers, and video cameras to connect to the Internet and share information with each other, creating a network.
The term "Wi-Fi" is an industry term for a type of wireless local area network (LAN) protocol based on the IEEE 802.11 networking standard.\cite{WiFiAllianceDiscover,CiscoWhatIsWiFi}

Using Wi-Fi, it is possible to detect the presence of a person carrying a Wi-Fi enabled device in a room.
This can be achieved by measuring the signal strength of the Wi-Fi signal received by the device.
\citeauthor{longoAccurateOccupancyEstimation2019} presents a method to estimate the number of people in a room using Wi-Fi by capturing probe request frames from Wi-Fi enabled devices.
By having Wi-Fi enabled devices broadcast probe request frames periodically, a sniffer device can capture these frames and count the number of unique MAC addresses observed. \cite{longoAccurateOccupancyEstimation2019}

The results show that Wi-Fi was fairly accurate in estimating the number of people in a room, however doing so was found to be quite expensive.
Furthermore, because of the high range of WiFi (50 to 150 m), sniffer devices are able to detect devices far outside the area of consideration.
Consequently, one would have to rely fully on the Received Signal Strength indicator (RSSI) to determine whether a device is in the room or not.
In addition, manufactures often transmit probe requests with randomized, bogus MAC addresses to prevent trivial tracking of devices, thus protecting the privacy of the user.
This makes it difficult to determine whether a device is in the room or not, as the MAC address of the device is not always the same.\cite{longoAccurateOccupancyEstimation2019}