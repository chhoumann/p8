\subsection{Wi-Fi} \label{sec:wi-fi}
Wi-Fi is a wireless networking technology allowing devices to communicate with each other and access the Internet without the need for cable connections. 
Wi-fi is based on the IEEE 802.11\cite{iEEE80211} standards and uses radio waves to transmit data between devices.
Wi-Fi allows devices such as computers, smartphones, printers, and video cameras to connect to the Internet and share information with each other, creating a network.
The term "Wi-Fi" is an industry term for a type of wireless local area network (LAN) protocol based on the IEEE 802.11 networking standard.\cite{WiFiAllianceDiscover,CiscoWhatIsWiFi}

It is possible to detect the presence of a person carrying a Wi-Fi enabled device in a room.
This can be achieved by measuring the signal strength of the Wi-Fi signal received by the device.
Multiple approaches using this measured signal exist. 
One possible way of doing this is by measuring the disturbances in signal strength received by devices and infer wether this disturbance is due to human movements. 
This was shown to be accurate when used for presence detection in meeting rooms with an accuracy of 96\%. \cite{wifi_presence_detection_df}
Furthermore,\citeauthor{longoAccurateOccupancyEstimation2019}\cite{longoAccurateOccupancyEstimation2019} presents a method to estimate the number of people in a room using Wi-Fi by capturing probe request frames from Wi-Fi enabled devices.
By having Wi-Fi enabled devices broadcast probe request frames periodically, a sniffer device can capture these frames and count the number of unique MAC addresses observed.
They show that Wi-Fi can be used to accurately estimate the number of people in a room.
However, doing so was found to be quite expensive.
Furthermore, because of the high range of WiFi (50 to 150 m), sniffer devices are able to detect devices far outside the area of consideration.\cite{longoAccurateOccupancyEstimation2019}

As presence detection using wifi is fully reliant on the measured Received Signal Strength indicator (RRSI) the systems are in turn vulnerable to environmental events causing fluctuations in the signal strength.
This is especially the case for low-powered wireless sensor networks, where competing wifi-networks are deployed in close proximity.\cite{competing_wifi}
In addition, manufactures often transmit probe requests with randomized, non-existing MAC addresses to prevent trivial tracking of devices, thus protecting the privacy of the user.
This makes it difficult to determine whether a device is in the room or not, as the MAC address of the device is not always the same.\cite{longoAccurateOccupancyEstimation2019}