\section{Choice of technology}
In this section we will elaborate on the decision for using Bluetooth for presence detection by examining the considerations that led us to this choice, by comparing the benefits and drawbacks of each technology to Bluetooth.
A key factor in this decision, in addition to the advantages that Bluetooth presents in our use case, was the preference of our stakeholders. During the interview process, they made it clear that they preferred using radio technology citing their familiarity with such technologies. Besides their preference, there were other considerations with alternative technologies that made them less ideal.



% We decided to use Bluetooth instead of object detection and environment sensing because:
% - Stakeholder prefers radio technology
% - Inconsitent accuracy (see longoAccurateOccupancyEstimation2019)
% - Affected by unpredictable opening and closing of doors and windows (see longoAccurateOccupancyEstimation2019)
% - Difficult to acquire
% - Privacy concerns (cameras, object detection, etc.)
% - Costly to setup and sensitive to light levels (video-based) 
% - Sensitive to obstrutions in room

% We decided to use Bluetooth instead of WiFi because:
% - Specifically requested by stakeholder (often used in offices)
% - Low power consumption
% - Low cost
% - Very cheap (see longoAccurateOccupancyEstimation2019)
% - Works indoor and outdoor
% - Easy to setup
% - High availability
% - In contrast to MAC address spoofing, Bluetooth devices can be identified by their Bluetooth address, which is unique to each device.
% - Ideally, we should combine Wi-Fi and Bluetooth, however we do not need such high accuracy (see longoAccurateOccupancyEstimation2019), and it would be far more expensive in terms of hardware.
% - We need local detection, however WiFi offers far greater range at the cost of accuracy, which is not needed for our application.

% Some issues with radio that we should keep in mind during development:
% - Privacy concerns
% - Not always available (based on the number of available handheld devices vs people - we cannot assume that there is a 1:1 correlation between people and number of devices). 