\section{Exploring Designs} % generel hci
Buxton\cite{BUXTON2007135_skething,BUXTON2007139_prototyping} describes how to move from an exploratory analysis of different designs through sketching to interactive product refinement using prototypes. 
Sketching is the act of ideation and capturing key concepts of the design in a cheap and easy-to-make artifact. 
It is important to note that the artifact itself is a byproduct of the process, while the goal of the activity lies within the process itself.\cite{BUXTON2007135_skething}
The goal is to understand user needs and develop a design concept that both the designer and user find agreeable in terms of requirements and user experience. 
When sketching, one must put aside assumptions and preconceptions about the product being designed.
This helps the designer discover new ideas for the product by not being constrained by an existing understanding of requirements, user needs, and expert knowledge.
Sketching techniques often include the participation of potential users of the product being developed.
This helps the designer focus efforts on designing a system that fulfills user needs.
This is known as user-centered design (UCD). \cite{user-centred-design}
Techniques within UCD include user sketching, prototype evaluation, and usability experiments involving users.

The following sections detail techniques for creating a prototype, and how such a prototype can be used when performing a usability evaluation.