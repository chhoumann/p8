\section{Designing a product} % generel hci
When establishing a design for a product, one often begin with exploring initial thoughs and ideas for a design, only to then present the design to the user, make adjustments, and repeat the process. 
This can be described in process called the design funnel.\cite{BUXTON2007135_skething}

Buxton\cite{BUXTON2007135_skething,BUXTON2007139_prototyping} describes how to move from exploratory analysis of different designs through skething to interactive, product refinement using prototypes.   
Sketching is the act of ideation and capturing key concepts of the design in a cheap and easy-to-make artifact. 
It is important to notice that the artifact itself is but a byproduct of the process, while the goal of the activity lies within the process itself.\cite{BUXTON2007135_skething}
The true goal is to understand user needs and develop a design concept that both designer and user finds aggreable in terms of requirements and user experience. 
When skething, one must put aside  assumptions and preconceptions about the product one is designing.
This helps the designer discover new ideas for the product due to not being constrained by existing  understanding of requirements, user needs and expert knowledge. %kilde? 
Sketching techniques often include participation of users of the developing product.
This helps the designer focus efforts on designing a system that suits the user and fulfills their needs.
This is known as user-centred design (UCD). \cite{user-centred-design}
Techniques within UCD include user sketching, prototype evaluation, and usability experiments involving users.

In the following sections this report detail techniques for creating a prototype, and how such prototype can be used when performing a usability evaluation.