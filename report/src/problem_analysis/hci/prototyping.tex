\section{Prototyping} %specifict prototyping
A prototype is an instantiation of a design concept meant for describing, refining, testing, and depicting the design.\cite{BUXTON2007139_prototyping}

When designing prototypes, the emphasis is not on the discovery of new knowledge and user needs, but to examine whether or not the proposed design constitutes a suitable hypothesis for the design of the product\cite{nielsen-norman-prototype-low-vs-high,BUXTON2007139_prototyping}
That is, the design must allow for pleasurable interaction and suit user needs to a degree deemed adequate by both user and designer.

In general, two categories of prototypes exist: High Fidelity (hifi) and Low Fidelity (lofi).
Lofi prototypes are typically limited in their function and are used to emphasize concepts within the design, alternative design solutions, and model the general interaction with the artifact or system\cite{low-vs-high-fidelity-prototype}.
These prototypes are helpful when performing ideation\cite{nielsen-norman-ideation} or when establishing user capabilities in terms of interaction with the system \cite{usefullness-of-different-prototypes,low-vs-high-fidelity-prototype}.
Lofi prototypes should be used early in the design process to establish the overall direction for a product design. 
Therefore, the creation of these prototypes and the following evaluations using them should not be a time-consuming or expensive process. \cite{usefullness-of-different-prototypes,low-vs-high-fidelity-prototype} 
Lofi prototypes are not ideal when evaluating user interaction as they are used to present a design idea to the user, and thus are not interacted with by the user\cite{low-vs-high-fidelity-prototype}.

Unlike lofi prototypes, hifi prototypes provide near-complete functionality and interactivity with a design closely representing the final product.
They enable users to interact with buttons, icons, and input fields.
However, due to their detailed and interactive nature, constructing hifi prototypes is more time-consuming than creating lofi prototypes. 
This makes hifi prototypes more suitable for examining users' interactions with the designed artifact, than exploring new design ideas. \cite{nielsen-norman-prototype-low-vs-high,low-vs-high-fidelity-prototype}



Since the stakeholder has already suggested ideas for a user interface design based on existing systems, we already have a direction for an initial product design. 
As we want to find a common denominator for the stakeholders' problems described in Section \ref{sec:understanding_the_problem} we need to understand how users interact with a product design resembling a potential solution.
To this end, a hifi prototype is more suitable than a lofi prototype \cite{low-vs-high-fidelity-prototype}.

\subsection{Developing a hifi prototype}
When designing a product multiple approaches can be taken. 
In this project, the stakeholder has suggested that we based our design on existing solutions.
Therefore, we endeavor towards a design strategy where we base the design on the current state-of-the-art systems for meeting-management and preferences of the stakeholder.

To develop the prototype, the sketching tool Figma is used \cite{Figma}.
Figma can be used to design both high- and low-fidelity prototypes, which enables the designer to easily create a prototype that allows the user to interact with the system.
Figma contains a rich set of user interface components that can be used and modified when creating the prototype.
When a user interacts with the design, by, for instance, clicking an object, the designer can enable a transition to a different view, thus simulating system behavior. 
Based on the initial interviews described in Section \ref{sec:understanding_the_problem}, the prototype must be suitable for use on a simple monitor, such as a tablet.
The prototype 

We develop a prototype highlighting two situations for meeting rooms.
Each situation depicts different states of a meeting room (occupied or available) which allows us to explore how users want to interact with the system in these different situations. 

Figure xx depicts the main view of a room is occupied.
%State of the art of products
%expert design and state of the art
%Presenting our product
