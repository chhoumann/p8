\section{Prototyping} %specifict prototyping
A prototype is an instantiation of a design concept meant for describing, refining, testing and depicting the design.\cite{BUXTON2007139_prototyping}

When designing prototypes, emphasis is not on the discovery of new knowledge and user needs, but to examine whether or not the proposed design constitutes a suitable hypothesis of the design of the product\cite{nielsen-norman-prototype-low-vs-high,BUXTON2007139_prototyping}
That is, the design must allow for pleasureable interaction and suit user needs to a degree deemed adequate by both user and designer.

In general, two categories of prototypes exist: High Fidelity (hifi) and Low Fidelity (lofi).
Lofi prototypes are typically limited in their function and are used to emphasize concepts within the design, alternative design solutions, and modelling the genral interaction with the artifact or system\cite{low-vs-high-fidelity-prototype}.
These prototypes are helpful when performing ideation\cite{nielsen-norman-ideation} or when establishing user capabilities in terms of interaction with the system \cite{usefullness-of-different-prototypes,low-vs-high-fidelity-prototype}.
Lofi prototypes should be used early in the design process to establish the overall direction for a product design. 
Therefore, the creation of these prototypes and the following evaluations using them should not be a time consuming or expensive process. \cite{usefullness-of-different-prototypes,low-vs-high-fidelity-prototype}  
Lofi prototypes are not ideal when performing evaluation of user interaction as they are used to present a design idea to the user, and thus is not interacted with by the user\cite{low-vs-high-fidelity-prototype}.

Unlike lofi prototypes, hifi prototypes provide near-complete functionaliy and interactivity with the a design closely represenging the final product.
They enable users to interact with buttons, icons, and input fields.
However, due to their detailed and interactive nature, constructing hifi prototypes is more time consuming than creating lofi prototypes. 
This makes hifi prototypes more suitable for examining users' interactions with the designed artifact, than exploring new design ideas. \cite{nielsen-norman-prototype-low-vs-high,low-vs-high-fidelity-prototype}



Since the stakeholder has already suggested ideas for a user interface design based on existing systems, we already have a direction for an initial product design. 
As we want to find a common denominator for the stakeholders' problems described in Section \ref{sec:understanding_the_problem} we need to understand how users interact with a product design resembling a potential solution.
To this end, a hifi prototype is more suitable than a lofi prototype \cite{low-vs-high-fidelity-prototype}.

\subsection{Developing a hifi prototype}
%State of the art of products
%expert design and state of the art
%Presenting our product
