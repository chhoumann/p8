\section{Evaluation results}\label{sec:usability_evaluation_results}
This section details the most relevant results of the usability test described in Section \ref{sec:prototype_test_plan}.
A transcript of each usability test can be found in Appendix \ref{appendix:usability_transcript}.

One of the problems described in Section \ref{sec:understanding_the_problem} explains that it is tedious to modify ongoing meetings. 
During the test, it was found that the prototype was received positively in this regard. 
Users found the interaction with and information on the prototype intuitive. 
They understood the different situations depicted on the prototype and understood how to interact and complete the test tasks described in Appendix \ref{appendix:usability_tasks} without issues.
During the debriefing, it was mentioned that the addition of being able to extend and modify meetings using a tablet would be an improvement to their current workflows.
However, it was speculated whether other approaches to making these changes to meetings could be done. 
The prototype focuses on the integration with existing systems already being used by the stakeholder and relies on user interaction for managing the meetings.
Since both the prototype and their current system is fully dependent on users correctly managing the meetings, some aspect of the problematic situation cannot be solved by this product. 
For instance, ending meetings early still relies on users remembering to register this is the system. 
Therefore, a solution should implement functionality able to automize necessary aspect of meeting management, minimizing reliance on users.
The feasibility of  automatic handling of one or more of the problematic situations should therefore be investigated.