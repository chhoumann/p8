\section{Results}\label{sec:usability_evaluation_results}
This Section details the most relevant results of the usability test described in Section \ref{sec:prototype_test_plan}.
A transcript of each usability test can be found in Appendix \ref{appendix:usability_transcript}.

One of the problems described in Section \ref{sec:understanding_the_problem} explains that it is tedious to modify ongoing meetings. 
During the test, it was found that the prototype was received positively in this regard. 
Users found the interaction with and information on the prototype intuitive. 
They were able to understand the different situations depicted on the prototype and understood how to interact and complete the test tasks described in Appendix \ref{appendix:usability_tasks} without issues. 
Furthermore, during the debriefing, it was mentioned that the addition of being able to extend and modify meetings using a tablet would be an improvement to their current workflows.
However, it was speculated whether other approaches to making these changes to meetings could be done. 
The prototype focused on usability issues by using a system based on similar systems and user stories from the stakeholders. 
This system focuses on the integration with existing systems already being used by the stakeholder and relies on user interaction for editing meetings. 
Therefore, technical approaches to automatically modifying and registering meetings were not considered. 
One user suggested that alternative approaches to ending meetings automatically should be considered, as people might forget to end meetings manually.
Therefore we examine methods for automatically detecting whether a room is in use.

