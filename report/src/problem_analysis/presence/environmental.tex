\subsection{Presence detection using $\text{CO}_{2}$ concentration}\label{sec:presence_env_co2}
As illustrated in the example above, it is possible to use $\text{CO}_{2}$ levels to detect presence in a room.
This approach uses $\text{CO}_{2}$ sensors to detect the presence of a person in a room, and is based on the assumption that the presence of a person will increase the $\text{CO}_{2}$ concentration in the room.\cite{gruberCO2SensorsOccupancy2014}

However, there are several causes of concern with this approach. Some notable ones are:
\begin{itemize}
    \item Ventilation flow rate matters as it affects the $\text{CO}_{2}$ concentration in the room. However, in some areas, this could be outside of the control of the user.
    \item Similarly, opening a door or window will also affect the $\text{CO}_{2}$ concentration in the room.
    \item There is a time delay between the presence of a person and a noticeable increase in $\text{CO}_{2}$ concentration. The more people that are present, the shorter the time delay will be.
\end{itemize}
\cite{gruberCO2SensorsOccupancy2014,longoAccurateOccupancyEstimation2019}

Various studies have been conducted to determine the accuracy of $\text{CO}_{2}$ sensors for occupancy detection. 
\citeauthor{longoAccurateOccupancyEstimation2019}\cite{longoAccurateOccupancyEstimation2019} summarizes some of these and show the accuracy of $\text{CO}_{2}$ measurements for occupancy detection is between 50\% and 99\%, depending on task and amount of occupants to estimate.