\subsection{Video-based detection} \label{sec:video-based-detection}
Cameras are a common technology used to detect the presence of humans in a room.
There are many different approaches to how one might use a camera for presence detection.\cite{FUERTES2022103473}

One of the primary methods for presence detection is the use of machine learning techniques.
These techniques work by training a machine learning model, such as a Convolutional Neural Network (CNN).
The resulting model can then be used to analyze images and detect objects in them.
For example, \citeauthor{FUERTES2022103473}\cite{FUERTES2022103473} used an omnidirectional camera to gather images of a room, and then trained a CNN to detect the presence of people in these images.
The training is done through point-based annotation of the images as opposed to the more common approach of using bounding box annotation.
Essentially, images are annotated with points denoting the location of the person in the image as opposed to drawing a bounding box around the person.
This approach is advantageous because it is faster to annotate images in this way, but it is also far more accurate at representing the position of a human on images taken with an omnidirectional camera, which are often distorted or skewed.\cite{FUERTES2022103473}

Motion detection is another commonly used to detect the presence of people in a room, and is defined by \citeauthor{ANANDAN1988347}\cite{ANANDAN1988347} as "\textit{the task of detecting movement of image elements across the image plane}".
In other words, motion detection is a broad term that can refer to a variety of different methods.
For example, motion detection can be achieved by comparing two images of the same scene, and detecting the differences between them\cite{granath_detecting_nodate}.

Alternatively, one may also utilize passive infrared (PIR) sensors to detect motion.
PIR sensors are able to detect motion by measuring the infrared radiation emitted by objects in the room.\cite{Deiana2014}

Many other methods exist, and the choice of method will ultimately depend on the specific application and system requirements. 
Regardless of the method, a number of key considerations apply when using cameras for presence detection, including the sensitivity of the devices to lighting, obstructions, movement, temperature, and detection range and area.
For example, if the device is obstructed by some object, it may be unable to gather useful data.
For cameras specifically, consider how it may not be able to get a clear image of a room if the lighting is too dark or too bright.
Most current systems also use standard perspective cameras with a narrow field of view that can only detect people in a limited area in front of the camera\cite{FUERTES2022103473}.
This, combined with positioning and orientation of the camera, can be detrimental to the quality of the data.\cite{granath_detecting_nodate, tang_occupancy_2020}

These limitations can increase the complexity of the system, as one would have to take these factors into account during system design and development, and when deciding which devices to use as well as where to place them.
Some limitations can be overcome by using multiple sensors or cameras, but this futher increases the cost.\cite{FUERTES2022103473}

Using cameras for presence detection also raises privacy concerns.
Being under surveillance by a camera is problematic, as it can be seen as a violation of privacy.
Consequently, applications employing cameras for presence detection must include a privacy policy that outlines how the data is collected, stored, and used, which all employees must also agree to.
These requirements add another layer of complexity to the system, as the privacy policy must be implemented and enforced by the system and the system owner.
Moreover, because of these privacy concerns, acceptance and subsequent use of the system may be limited.\cite{granath_detecting_nodate, tang_occupancy_2020, PrivacyPreservingSensor}