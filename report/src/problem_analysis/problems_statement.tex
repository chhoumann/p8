\section{Problem statement}\label{sec:problem_statement}
In Section \ref{sec:understanding_the_problem} it was established that the stakeholders have several issues regarding the management of meetings. 
To resolve these issues, the stakeholders proposed a system defined by several user stories detailed in Appendix \ref{appenix:user_stories}. 
To examine whether such a system would solve the stakeholders' issues, a hifi prototype was developed using Figma\cite{Figma} by following methods from user-centred design.
The prototype was tested using an approach to summative usability testing defined by \citeauthor{lazar2005web}~\cite{lazar2005web}. 
The result of this test was that, while many issues regarding user usability could be resolved by a system based on the prototype, some remain. 
One issue remaining is that meeting rooms might still be unused, as the system relies on users manually registering that meetings have concluded early.
To alleviate these issues, a technology that can automatically identify when a room is unused could be deployed. This leads to the following problem statement:
\begin{problem_statement}
    Is it possible to locate a person accurately enough to determine whether they are occupying a meeting room?
\end{problem_statement}

