\section{Understanding the problem}\label{sec:understanding_the_problem} % What constitutes a solution for 
To establish knowledge about the domain and issues with the current system for managing meeting arrangements, two interviews with the stakeholders were conducted. 
The goals of these interviews were twofold: 
First and foremost, we wanted to understand the current issues the stakeholders are facing. 
To accommodate this, an unstructured interview was used to explore the domain and problem\cite{benyon2013designing}. 
This form of interview is beneficial when exploring the domain, as unexpected findings and understanding is often found when exploring the situation with the users\cite{benyon2013designing} 
Secondly, we wanted to understand what the stakeholder sees as minimum requirements for a potential solution to their problem. 
To reach this understanding, a semi-structured interview was conducted, after the first interview.
Due to not understanding all of the specifics of the stakeholders' probles, we could not create close-ended questions for a structured interview, and thus chose to use a semi-structured interview focusing on open-ended questions. This allowed us to ask complementary questions to clarify answers \cite{InterviewsNHS}.
This interview was based on ideas for functionality and UI provided to us by the stakeholder during the initial contact.


During the initial interview, the stakeholders provided us with insight into what their current needs are, and why their current system does not satisfy these needs.

First and foremost, the stakeholder has issues with arranging meetings throughout their day.
When employees of Mariendal IT schedule meetings, they sometimes experience problems with not having or not being able to locate available rooms.
The unavailability is often due to the meeting organizer not registering that a meeting has concluded early.
This leads to meeting rooms being unattended while still being marked as in use in their calendar system.
Furthermore, the stakeholder will sometimes have unscheduled visits from clients, and will thus need a place to discuss business impromptu. 
Locating which rooms are available requires the scheduler to use their computer to assess each room's availability.
Alternatively, the scheduler will visit each meeting room to assess whether it is currently available and then use their computer to book the room if it is available for long enough.
During the interview, they mention that a simple screen, like a tablet, showing the room's current availability could be a partial solution to this.

Secondly, registering new meetings requires the user to use their computer and open proprietary software to schedule the meeting and reserve a room.
Making changes to an existing meeting, such as extending the duration of the meeting, can only be done by finding the meeting in their current system, and editing the details of the meeting.
This makes it tedious and time-consuming to extend meetings, as the organizer has to leave the room to find a computer and register the changes.

Thirdly, making arrangements other than reserving rooms can be time-consuming and annoying. 
It is sometimes necessary to bring IT equipment to meetings to showcase them to customers.
This means that it is necessary to plan and coordinate where and when equipment is in use to ensure that it is available for meetings. 
Similarly, it is customary to provide refreshments and snacks during the meeting, which must also be arranged. 
Currently, this is done by reaching out to the cafeteria via phone or email.
This adds yet another task that needs to be solved to accommodate meetings.

Based on the stakeholders' issues and our understanding of the problem, a set of user stories were developed and prioritized using MoSCoW analysis \cite{DEBbook}.
These can be seen in Appendix \ref{appendix:user_stories}.

Having established and performed an initial examination of the stakeholders' obstacles toward a useful system through interviews, we now partake in the process of trying to find a common denominator for these problems.
This will be done by developing a product design and examining user interactions and opinions of this product design.