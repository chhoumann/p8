\pdfbookmark[0]{English title page}{label:titlepage_en}
\aautitlepage{%
  \englishprojectinfo{
    P8
  }{%
    Mobility
  }{%
    Spring Semester 2023 %project period
  }{%
    cs-23-sw-8-01 % project group
  }{%
    %list of group members
    Christian Bager Bach Houmann\\
    Daniel Overvad Nykjær\\
    Ivik Lau Dalgas Hostrup\\
    Marco Klaustrup Justesen\\
    Patrick Frostholm Østergaard\\
    Rasmus Høyer Hansen
  }{%
    %list of supervisors
    Christian Tovgaard Aebeloe
  }{%
   May, 2023
  }%
}{%department and address
  \textbf{Computer Science}\\
  Aalborg University\\
  \href{http://www.aau.dk}{http://www.aau.dk}
}{% the abstract
This report presents an in-depth investigation into the application of Bluetooth Low Energy (BLE) and fingerprinting for presence detection.
The study commenced with an analysis phase to understand the needs of a company for presence detection, leading to the exploration of multiple localization technologies.
Among them, it was found that BLE, combined with fingerprinting, was a promising approach.
The research further delves into the transformation of this localization problem into a classification problem using a kNN-based algorithm.
A C\# application was developed to evaluate this method for occupancy classification.
The results of the evaluation demonstrates the efficacy of BLE and fingerprinting for reliable presence detection.
}{% keywords
}

% \cleardoublepage
% {\selectlanguage{danish}
% \pdfbookmark[0]{Danish title page}{label:titlepage_da}
% \aautitlepage{%
%   \danishprojectinfo{
%     Rapportens titel %title
%   }{%
%     Semestertema %theme
%   }{%
%     Efterårssemestret 2010 %project period
%   }{%
%     XXX % project group
%   }{%
%     %list of group members
%     Forfatter 1\\
%     Forfatter 2\\
%     Forfatter 3
%   }{%
%     %list of supervisors
%     Vejleder 1\\
%     Vejleder 2
%   }{%
%     1 % number of printed copies
%   }{%
%     \today % date of completion
%   }%
% }{%department and address
%   \textbf{Elektronik og IT}\\
%   Aalborg Universitet\\
%   \href{http://www.aau.dk}{http://www.aau.dk}
% }{% the abstract
%   Her er resuméet
% }}
%