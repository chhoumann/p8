\section{Implementing Multi-Factor Authentication}
% **Train of thought during the implementation, for reference:**
% - Need to implement this user story: As a user, I want the platform to use MFA login for both users and the admin page, to ensure that only authorized individuals can access and use the system.
% - System currently has no auth. Need to implement that as well.
% - Customer wants integration with Outlook. Outlook uses Microsoft. We should integrate with Azure for authentication.
% - Spent a lot of time on this. Unfortunately, nothing really panned out, as the integration to our platform's Auth framework has misaligned input/output schemas. So we had to change our authentication system from NextAuth.js to Clerk. Clerk is an identity management platform that simplifies and secures user authentication by providing built-in support for features such as MFA. The migration to Clerk laid the foundation for adding MFA functionality to the platform. Clerk enables using Microsoft with your own Azure application, so this accomplishes what we need. Just need to integrate now.
In this section, we discuss the implementation of multi-factor authentication (MFA)\cite{multifactor} for users of the application, as well as integration with Microsoft Outlook\cite{microsoftOutlook} using Azure\cite{azure} for authentication.
The aim is to ensure that only authorised individuals can access and use the system, in line with the user story: "\textit{As a user, I want the platform to use MFA login for both users and the admin page, to ensure that only authorized individuals can access and use the system.}"

\subsection*{Initial Considerations and Challenges}
Initially, the platform lacked an authentication system, which needed to be implemented.
Furthermore, the customer requested integration with Outlook, which is a Microsoft product\cite{microsoftOutlook}.
Consequently, we considered integrating with Azure to fulfill the customer's requirements.
This would afford us the ability to use Microsoft's authentication system with our own Azure application.

We also knew that the platform should be web-based.
Due to prior experience, the Scrum team settled on using Next.js\cite{nextjs} as framework for this.
To implement authentication, we explored the possibility of using NextAuth.js\cite{nextAuth} to integrate with Azure.

However, after spending significant time on this approach, we encountered misaligned input/output schemas between our NextAuth.js and the integration with Azure.
As a result, we had to reconsider our approach and explore alternative authentication systems.

After evaluating several options, we decided to move our authentication system from NextAuth.js to Clerk\cite{clerk}.
Clerk is a user management platform that simplifies and secures user authentication by providing built-in support for features such as MFA.
Migrating to Clerk laid the groundwork for adding MFA functionality to the platform.

\subsection*{Implementation}
Clerk enables the use of Microsoft authentication through integration with Azure applications with Microsoft Azure Active Directory\cite{clerkMicrosoft}.
This feature allows us to meet the customer's request for Outlook integration while maintaining a secure and reliable authentication system.

The Clerk platform provides a simple API that allowed us to integrate with the platform and implement MFA.
In practice, the implementation involved installing a package, adding the relevant components and managing the environment secrets.
We were then able to use the Clerk API to authorise users and protect routes.

In conclusion, the changes to the application's authentication system have successfully implemented MFA for users, as well as integrating with Azure for Microsoft Outlook authentication.
This approach addresses the challenges faced during the initial integration attempt and meets the requirements outlined in the user story.



