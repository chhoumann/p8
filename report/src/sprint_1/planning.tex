\subsection*{Planning}
The sprint planning phase started with an analysis of the must-have user stories. 
This was done to create an initial prioritization based on which were most critical to development.
Therefore for the initial sprint, we chose to prioritize the user stories relating to Outlook authorization and log-in, as these were deemed necessary for the remaining user stories. 
Thus the chosen user stories for this sprint and their granulated tasks are as follows:
\begin{itemize}
    \item As a user, I want the platform to use MFA login for both users and the admin page, to ensure that only authorized individuals can access and use the system.
    \begin{enumerate}
        \item Create wire-frame
        \item Create mockups
        \item Outlook authentication
        \item Login functionality
    \end{enumerate}
    \item As a user, I want to be able to book a meeting room through a website and view it in Outlook afterward, so that I can use the platform that is most convenient for me.
    \begin{enumerate}
        \item Create wire-frame
        \item Create mockups
        \item Basic meeting booking functionality
    \end{enumerate}   
\end{itemize}

Based on the chosen user stories as well as their granulated tasks the goal for this sprint is to develop a mockup to gather feedback as well as a website that supports basic meeting booking functionality through Outlook.




% What do we we prioritize and why?
% Show backlog? Granulering / små tasks?
% Sprint goal & backlog