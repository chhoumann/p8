\section{Prototyping} %specifict prototyping
A prototype is an instantiation of a design concept meant for describing, refining, testing and depicting the design. \cite{BUXTON2007139_prototyping}

When designing prototypes, emphasis is not on the discovery of new knowledge and user needs, but to examine whether or not the proposed design constitutes a suitable hypothesis of the design of the product \cite{nielsen-norman-prototype-low-vs-high,BUXTON2007139_prototyping}
That is, the design must allow for pleasureable interaction and suit user needs to a degree deemed adequate by both user and designer.

In general, two categories of prototypes exist: High Fidelity (hifi) and Low Fidelity (lofi).
Lofi prototypes are typically limited in their function and are used to emphasize concepts within the design, alternative design solutions, and modelling the genral interaction with the artifact or system \cite{low-vs-high-fidelity-prototype}.
These prototypes are helpful when performing ideation %kilde paa ideation%
or when establishing user capabilities in terms of interaction with the system \cite{usefullness-of-different-prototypes,low-vs-high-fidelity-prototype}. 
Lofi prototypes should be used early in the design process. 
Therefore, the creation of these prototypes and the following evaluations  using them should not be a time consuming or expensive process. \cite{usefullness-of-different-prototypes,low-vs-high-fidelity-prototype}  


Hifi prototypes can be used to xxxx
can be used to xxxxxxxxxxx

`'


\subsection{Developing the prototype}
Development of prototype