\section{Phase one: Data collection}\label{sec:first_phase}
As the quality of the system is reliant on the accuracy and precision of the map during this phase it is critical to create a map of high quality.
Therefore, steps must be taken to ensure this quality. 
\citeauthor{improving_indoor_localization}~\cite{improving_indoor_localization} suggests that measurements be taken by systematically partitioning the area in a grid and then performing measurements for each cell of the grid for n-seconds. 
We propose a design where, instead of having a fixed collection time, a fixed minimum number of measurements is received from each beacon.
We do this, as beacons broadcast on a fixed time-schedule \cite{apple2023ibeacon}.
This is done to ensure that all beacons have a representative set of measurements with low standard deviation when constructing the map.
To ensure a low standard deviation, we use the following procedure.
Data is collected from each beacon until a minimum number of data points are collected, and a stable standard deviation within a satisfactory threshold is reached.
As fluctuations can cause large deviations in the collected data, the data collection for a beacon is halted as soon as the prior condition is satisfied.
The average RSSI value from the collected measurements from the beacons is used to create entries for the map.
During the measurement collection each cell is labelled as being either inside or outside of the meeting room.
This process is repeated for each cell in the grid to construct the entries for the map. 


