\section{Introduction: System Design}
In Chapter \ref{chap:presence_intro} technologies that can be used for presence detection were described. 
Technology selection criteria based on \citeauthor{presence_ble_review}~\cite{presence_ble_review} were developed and identified Bluetooth as the most suitable technology for a presence system satisfying the user problems described in Section \ref{sec:understanding_the_problem}.

Chapter \ref{sec:presence_detection_using_ble} then details two approaches to using Bluetooth for presence detection. 
We found that a fingerprinting approach using kNN is suitable for occupancy detection. 
We argue that the approach can easily be modified with additional labels when collecting data in the first phase of the approach, as described in section \ref{sec:fingerprinting}.

In the following sections we will first present a general overview of the application and how it supports the two phases of fingerprinting. 
Section \ref{sec:first_phase} establishes how we approach the collection of measurements for the first phase, followed by how this can be done using an App for smartphone. 
We describe how we approach the data collection, and how we accomodate the need for labelling of the collected data.
In Section \ref{sec:ble_implementation}, we discuss the key design decisions and implementation aspects that significantly influence this application.
Similarly, Section \ref{sec:knn_implementation} details how the second phase of fingerprinting is implemented in the application, and how a classification of \textit{present} or \textit{absent} is performed.



