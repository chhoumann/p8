\section{Phase two: Presence classification}\label{sec:knn_implementation}
The \texttt{Data Handler}, as shown in Figure~\ref{fig:implementation_diagram}, loads the data set from the JSON file and removes any duplicates to ensure data consistency.
Failing to eliminate duplicates could lead to erroneous classifications in the kNN computations.
This is because each duplicate data point would be equidistant from the target point and could potentially be counted as the k-nearest neighbors, even when non-duplicate points are available.

To classify, the \texttt{Data Handler} calculates the distance between the current RSSI measurements and the collected RSSI data points.
These distances are then used to find the k-nearest neighbors.
Once the nearest neighbors are found, the weighted number of neighbors inside the room are calculated and compared to a threshold.
Because the weight is normalized, this threshold must be between 0 and 1.
It can be adjusted in the UI using a slider in order to experimentally determine the best configuration.
If the weighted number of neighbors inside the room is greater than this threshold, the resulting classification is \textit{inside} the room, and otherwise it is classified as \textit{outside} the room.
The classification algorithm can be seen in code snippet~\ref{lst:knn-implementation}.

\begin{figure}[H]
\begin{lstlisting}[
	language=CSharp, 
	frame=single, 
	label={lst:knn-implementation},
	escapeinside={(*}{*)},
	caption={C\# Implementation of the K-Nearest Neighbors Algorithm with Weighted Voting}, 
	captionpos=b] 
public ClassLabel Classify( (*\label{line:classify}*)
	int[] rssis, 
	IReadOnlyList<DataPoint> rssiDataPoints
) {
	DataPointDistance[] distances = CalculateDistances(
		rssis, 
		rssiDataPoints
	);

	DataPointDistance[] kNearestNeighbors = GetKNearestNeighbors( (*\label{line:GetKNearestNeighbors}*)
		distances,
		k
	);

	double weightedNumNeighborsInsideRoom = CalculateWeight( (*\label{line:weighted-num-neighbors-inside-room}*)
		kNearestNeighbors,
		ClassLabel.Inside
	);

	return weightedNumNeighborsInsideRoom > threshold 
		? ClassLabel.Inside 
		: ClassLabel.Outside;
}
\end{lstlisting}
\end{figure}

The \texttt{CalculateWeight} method, seen in code snippet~\ref{lst:knn-implementation} on line \ref{line:weighted-num-neighbors-inside-room}, calculates the weighted number of neighbors matching the provided target class label using the k-nearest neighbors. 
The \texttt{GetKNearestNeighbors} method on line \ref{line:GetKNearestNeighbors}, and the \texttt{CalculateWeight} method on line \ref{line:weighted-num-neighbors-inside-room}, are implementations of the approach described in Section~\ref{sec:phase2_classification}.

In the \texttt{Classify} method, as seen on line \ref{line:classify} in code snippet~\ref{lst:knn-implementation}, we use the resulting weight to determine whether the current point is inside or outside the room by comparing it to a threshold.
The threshold value can be adjusted to meet specific requirements for the application. A higher threshold will result in a stricter classification for \textit{inside} the room, while a lower threshold will make it more lenient.