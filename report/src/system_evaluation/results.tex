\section{Results}\label{sec:experiment_results}
%describe in terms of criteria for using bluetooth. `'
Having created a map that can be used to classify received measurements and having collected information about these classifications, we will now summarizes and comment on the results. 

\begin{table}[H]
    \scalebox{0.8}{
    \begin{tabular}{|c|llllllllllll|}
    \hline
    \multicolumn{1}{|l|}{K-Value} & \multicolumn{12}{c|}{Position} \\ \hline
    \multicolumn{1}{|l|}{}        & \multicolumn{1}{c|}{a}     & \multicolumn{1}{c|}{b}    & \multicolumn{1}{c|}{c}     & \multicolumn{1}{c|}{d}     & \multicolumn{1}{c|}{e}     & \multicolumn{1}{c|}{f}     & \multicolumn{1}{c|}{g}     & \multicolumn{1}{c|}{h}    & \multicolumn{1}{c|}{i}    & \multicolumn{1}{c|}{j}    & \multicolumn{1}{c|}{k}     & \multicolumn{1}{c|}{l} \\ \hline
    2                             & \multicolumn{1}{l|}{10/10} & \multicolumn{1}{l|}{8/10} & \multicolumn{1}{l|}{10/10} & \multicolumn{1}{l|}{9/10}  & \multicolumn{1}{l|}{10/10} & \multicolumn{1}{l|}{8/10}  & \multicolumn{1}{l|}{10/10} & \multicolumn{1}{l|}{4/10} & \multicolumn{1}{l|}{6/10} & \multicolumn{1}{l|}{2/10} & \multicolumn{1}{l|}{5/10}  & 7/10                   \\ \hline
    3                             & \multicolumn{1}{l|}{10/10} & \multicolumn{1}{l|}{6/10} & \multicolumn{1}{l|}{10/10} & \multicolumn{1}{l|}{9/10}  & \multicolumn{1}{l|}{10/10} & \multicolumn{1}{l|}{10/10} & \multicolumn{1}{l|}{10/10} & \multicolumn{1}{l|}{7/10} & \multicolumn{1}{l|}{6/10} & \multicolumn{1}{l|}{7/10} & \multicolumn{1}{l|}{9/10}  & 10/10                  \\ \hline
    4                             & \multicolumn{1}{l|}{10/10} & \multicolumn{1}{l|}{8/10} & \multicolumn{1}{l|}{10/10} & \multicolumn{1}{l|}{10/10} & \multicolumn{1}{l|}{10/10} & \multicolumn{1}{l|}{10/10} & \multicolumn{1}{l|}{9/10}  & \multicolumn{1}{l|}{9/10} & \multicolumn{1}{l|}{8/10} & \multicolumn{1}{l|}{6/10} & \multicolumn{1}{l|}{10/10} & 10/10                  \\ \hline
    \end{tabular}}
    \caption{Evaluation results from Samsung Galaxy A53 showing the number of correct classifications out of a possible ten for each k-value}
    \label{lst:resultsPhone}
\end{table}

\begin{table}[H]
    \scalebox{0.8}{
    \begin{tabular}{|c|llllllllllll|}
    \hline
    \multicolumn{1}{|l|}{K-Value} & \multicolumn{12}{c|}{Position}                                                                                                                                                                                                                                                                                                                   \\ \hline
    \multicolumn{1}{|l|}{}        & \multicolumn{1}{c|}{a}     & \multicolumn{1}{c|}{b}    & \multicolumn{1}{c|}{c}     & \multicolumn{1}{c|}{d}     & \multicolumn{1}{c|}{e}     & \multicolumn{1}{c|}{f}     & \multicolumn{1}{c|}{g}     & \multicolumn{1}{c|}{h}    & \multicolumn{1}{c|}{i}    & \multicolumn{1}{c|}{j}    & \multicolumn{1}{c|}{k}    & \multicolumn{1}{c|}{l} \\ \hline
    2                             & \multicolumn{1}{l|}{10/10} & \multicolumn{1}{l|}{9/10} & \multicolumn{1}{l|}{10/10} & \multicolumn{1}{l|}{10/10} & \multicolumn{1}{l|}{10/10} & \multicolumn{1}{l|}{8/10}  & \multicolumn{1}{l|}{10/10} & \multicolumn{1}{l|}{2/10} & \multicolumn{1}{l|}{2/10} & \multicolumn{1}{l|}{3/10} & \multicolumn{1}{l|}{6/10} & 7/10                   \\ \hline
    3                             & \multicolumn{1}{l|}{10/10} & \multicolumn{1}{l|}{9/10} & \multicolumn{1}{l|}{10/10} & \multicolumn{1}{l|}{10/10} & \multicolumn{1}{l|}{10/10} & \multicolumn{1}{l|}{10/10} & \multicolumn{1}{l|}{9/10}  & \multicolumn{1}{l|}{4/10} & \multicolumn{1}{l|}{6/10} & \multicolumn{1}{l|}{2/10} & \multicolumn{1}{l|}{6/10} & 8/10                   \\ \hline
    4                             & \multicolumn{1}{l|}{10/10} & \multicolumn{1}{l|}{9/10} & \multicolumn{1}{l|}{10/10} & \multicolumn{1}{l|}{10/10} & \multicolumn{1}{l|}{10/10} & \multicolumn{1}{l|}{10/10} & \multicolumn{1}{l|}{10/10} & \multicolumn{1}{l|}{6/10} & \multicolumn{1}{l|}{6/10} & \multicolumn{1}{l|}{6/10} & \multicolumn{1}{l|}{8/10} & 8/10                   \\ \hline
    \end{tabular}}
    \caption{Evaluation results from Samsung Galaxy S20 showing the number of correct classifications out of a possible ten for each k-value}
    \label{lst:resultsPhoneS20}
\end{table}

The classification results of the experiments were gathered in the locations shown on Figure \ref{fig:experiment_room}. During the experiment, k-values of $2$, $3$, and $4$ were used.  
The results of the experiment can be seen in Tables \ref{lst:resultsPhone}, \ref{lst:resultsPhoneS20}, and  \ref{lst:resultsTablet}.
The tables shows the number of correct classifications. 
Table \ref{lst:resultsPhone_precision} summarizes the experiment's results, showing the percentage of true positive classifications for the different locations from Figure \ref{fig:experiment_room}.

On Table \ref{lst:resultsPhone_precision} see that a lower k-value results in a larger uncertainty in the classification.
However, it is worth noticing that for measuring points \textit{B} and \textit{J}  (see Figure \ref{fig:experiment_room}) a large uncertainty in classification is observed.
We suspect that this is a consequence of the glass pane next to the door having a much lower absorbation of radio frequencies than the door and the wall next to it. 
This low absorbation results in the two data points next to the pane having similar distances to both map values labelled as inside and some as outside, therefore giving uncertain classifications.

\begin{table}[H]
    \scalebox{0.8}{
    \begin{tabular}{|c|llllllllllll|}
    \hline
    \multicolumn{1}{|l|}{K-Value} & \multicolumn{12}{c|}{Position}                                                                                                                                                                                                                                                                                                                   \\ \hline
    \multicolumn{1}{|l|}{}        & \multicolumn{1}{c|}{a}     & \multicolumn{1}{c|}{b}    & \multicolumn{1}{c|}{c}     & \multicolumn{1}{c|}{d}     & \multicolumn{1}{c|}{e}     & \multicolumn{1}{c|}{f}     & \multicolumn{1}{c|}{g}     & \multicolumn{1}{c|}{h}    & \multicolumn{1}{c|}{i}    & \multicolumn{1}{c|}{j}    & \multicolumn{1}{c|}{k}    & \multicolumn{1}{c|}{l} \\ \hline
    2                             & \multicolumn{1}{l|}{10/10} & \multicolumn{1}{l|}{8/10} & \multicolumn{1}{l|}{10/10} & \multicolumn{1}{l|}{10/10} & \multicolumn{1}{l|}{10/10} & \multicolumn{1}{l|}{9/10}  & \multicolumn{1}{l|}{10/10} & \multicolumn{1}{l|}{0/10} & \multicolumn{1}{l|}{4/10} & \multicolumn{1}{l|}{5/10} & \multicolumn{1}{l|}{7/10} & 8/10                   \\ \hline
    3                             & \multicolumn{1}{l|}{10/10} & \multicolumn{1}{l|}{7/10} & \multicolumn{1}{l|}{10/10} & \multicolumn{1}{l|}{10/10} & \multicolumn{1}{l|}{10/10} & \multicolumn{1}{l|}{10/10} & \multicolumn{1}{l|}{10/10} & \multicolumn{1}{l|}{5/10} & \multicolumn{1}{l|}{5/10} & \multicolumn{1}{l|}{5/10} & \multicolumn{1}{l|}{8/10} & 8/10                   \\ \hline
    4                             & \multicolumn{1}{l|}{10/10} & \multicolumn{1}{l|}{9/10} & \multicolumn{1}{l|}{10/10} & \multicolumn{1}{l|}{10/10} & \multicolumn{1}{l|}{10/10} & \multicolumn{1}{l|}{10/10} & \multicolumn{1}{l|}{10/10} & \multicolumn{1}{l|}{6/10} & \multicolumn{1}{l|}{6/10} & \multicolumn{1}{l|}{7/10} & \multicolumn{1}{l|}{9/10} & 10/10                  \\ \hline
    \end{tabular}}
    \caption{Evaluation results from Lenovo TabM10 FHD Plus showing the number of correct classifications out of a possible ten for each $k$-value}
    \label{lst:resultsTablet}
\end{table}


Similarly, for all devices, when performing the classification in locations H, I, J, K, and L shows, the results are vastly different from the other locations.
These locations are all outside of the room. 
We suspect that this can be due to two things.
First and foremost, it might be due to \todo{something with physical obstacles}
Secondly, by having placed beacons only inside of the meeting room, and by having more classification data representing locations inside the meeting room, the classification results might be skewed towards \textit{inside} classifications. 

We also see that of all the devices (Tables \ref{lst:resultsPhone}, \ref{lst:resultsPhoneS20}, and  \ref{lst:resultsTablet}), \todo{figure out which one is more accurate} is more accurate. 
We suspect that this is due to xxxxx.

\begin{table}[H]
    \centering
    \begin{tabular}{|l|ll|}
    \hline
            & \multicolumn{2}{l|}{Accuracy}                      \\ \hline
    K-Value & \multicolumn{1}{l|}{Indoor}          & Outside      \\ \hline
    2       & \multicolumn{1}{l|}{200/210 (95.24\%)} & 68/150 (45.33\%) \\ \hline
    3       & \multicolumn{1}{l|}{200/210 (95.24\%)} & 96/150 (64.00\%) \\ \hline
    4       & \multicolumn{1}{l|}{205/210 (97.62\%)} & 118/150 (78.67\%) \\ \hline   
    \end{tabular}
    \caption{Accuracy of the classifications for different $k$-values}
    \label{lst:resultsPhone_precision}
\end{table}
As can be seen in Table \ref{lst:resultsPhone_precision}, the results show that the system can determine that the occupant is inside the room with high accuracy, even when a small $k$-value is selected. The approach is not as accurate when detecting that the occupant is outside the room, when small values of $k$ are selected.  