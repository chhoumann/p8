\section{Experiment design}\label{sec:experiment_design}
%Forsoegsopstilling, der inkludere en tegning af omraadet set oppefra med indtegning af beacons, objectern og hvor rigtige position for test person.
Experiments to evaluate the implemented system described in Chapter \ref{chap:system_design} are conducted in a a room measuring $4.75m \times 6.5m$.
In the room, a cabinet is placed between two tables, effectively dividing the room into two smaller areas. 
Each of the smaller areas are furnished with a table surrounded by six chairs, and whiteboards mounted on the walls.

On one end of each table is a large, rectangular window.
Furthermore, one table has a large TV screen placed near its window. 
Near the door of the room, an airconditioning controller and a large glass pane, acting as a window out into the common area, is located. 
Next to the door is placed a glass pane, acting as a window out into the common area. 
Lighting is built into the ceiling, thus no hanging light will interfere with the broadcasted signals. 
Outside the meeting room, is the aforementioned common area containing two high tables.
Each of these tables has a small couch and two chairs placed next to it.
The meeting room and its surrounding area is depicted in Figure \ref{fig:experiment_room}.

To create the fingerprinting-map, the room and the common area is partitioned into a $1m \times 1m$ grid using painter's tape.
Measurements for the map are taken where gridlines intersect.
This results in $6 \times 4$ measurements with a classification label of \textit{inside} being taken inside the room.
The common area outside of the room is partitioned in the same fashion, resulting in  $6 \times 2$ classification labels of \textit{outside}. 


\begin{figure}
    \centering
    \begin{subfigure}[b]{0.48\textwidth}
        \centering
        \includegraphics[width=\textwidth]{images/experiment_room.png}
        \caption{A sketch showing the room and the common area in which the experiments take place}
        \label{fig:experiment_room}
    \end{subfigure}
    \begin{subfigure}[b]{0.48\textwidth}
        \centering
        \includegraphics[width=\textwidth]{images/roomwithgrid.png}
        \caption{The gridwise partition of the experiment room and the common area (red)}
        \label{fig:room_partition}
    \end{subfigure}
    \begin{subfigure}[b]{0.48\textwidth}
        \includegraphics[width=\textwidth]{images/roomwithgridandmeasurements.png}
        \caption{The partitioned room decorated with locations where measurements are taken(green)}
        \label{fig:room_partition_measurements}
    \end{subfigure}
    \begin{subfigure}[b]{0.48\textwidth}
        \includegraphics[width=\textwidth]{images/gridwithbeacons.png}
        \caption{The partitioned room with four iBeacon BLE9 Bluetooth beacons. }
        \label{fig:room_partition_beacons}
    \end{subfigure}
    \caption{The room and common area in which the experiment takes place. Figures \ref{fig:room_partition} and \ref{fig:room_partition_measurements} shows how the area is partitioned and where measurements are taken. Figure \ref{fig:room_partition_beacons} shows the placement of beacons when performing the experiments. }
    \label{fig:allfiguresForTheGridPartition}
\end{figure}
The gridwise partition of the room and the common area is shown in Figure \ref{fig:room_partition}, while Figure \ref{fig:room_partition_measurements} depics the \textit{collection points} where measurements for the map is taken.
During the creation of the map, measurements are collected for each collection point until a standard deviation of under $6.5$ \todo{insert unit of standard deviation. What does 6.5 mean?} is achieved, with a minimum of $40$ measurements being required, as described in Section \ref{sec:data_collection_implementation}.
The map creation process was done during out-of-office hours to ensure that the datapoint collected to create the map was collected with a minimum of background signal noise.

During both the creation of the map and the classification, four iBeacon BLE9 Bluetooth beacons \cite{BluetoothiBeaconBLE9} are placed on top of the cabinet and in the three corners furthest away from the door. This is depicted on Figure \ref{fig:room_partition_beacons}.
The beacons in the corners of the room are placed in a height of $2.5m$, while the center beacon on top of the cabinet is placed in a height of $1.5m$.
The beacons placed near the windows are located on top of a whiteboard, while the beacon in the lower left corner of the room is held in place using painter's tape.

%Why are the beacons placed here?
By placing a beacon close to the door of the meeting room, one makes the classification system dependent on a high precision.
This can be understood by imagining a scenario where a person is located in a position (see Figure \ref{fig:experiment_setup}, measurement location B) only just inside the meeting room.
By having a beacon next by door, we risk that a slight fluctuation in the received signal strengh from the will result in the determined location of the person being that of the measurement point next to the door,immediatedly outside the room (see Figure \ref{fig:room_partition_measurements}).
For this particular system, we would rather have classifications skew towards being inside of the room as to not cancel meetings that are ongoing.  

%Mention the threshold and argue why its set to .5


%Explain the configuration of the grid


