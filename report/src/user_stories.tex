During requirement elicitation, we conducted a semi-structured interview with the project stakeholders.
This led to a collection of user stories.

Prior to the interview we sent open questions to the stakeholders such that they could reflect on the project and write down key ideas.
This led to a document of semi-prioritized points we would use as a basis for discussion during the interview.
When we encountered vague or unclear ideas, we would explore them further with the stakeholder during the interview.
Likewise, we clarified priorities and any conflicting elements.

After conducting the interview, we refined the ideas into user stories. These user stories were prioritized further using MoSCoW analysis.
These were sent to the stakeholder for feedback and to confirm mutual understanding.


\section{User Stories}
In this section, we present the user stories for the system that we have identified with the stakeholders.

\subsection{Must have}
\begin{itemize}
\item As a user, I want to be able to book a meeting room through Outlook/Outlook mobile and select it as a location, so that I can easily schedule my meetings and reserve a space.
\item As a user, I want to be able to book a meeting room through a website and view it in Outlook afterward, so that I can use the platform that is most convenient for me.
\item As a user, I want to be able to see who has booked a meeting room, so that I can avoid scheduling conflicts and ensure that the room is available when I need it.
\item As a user, I want there to be a screen at each meeting room displaying the name of the person who has booked it (without meeting details), and the ability to book the room immediately for an upcoming meeting.
\item As a user, I want the platform to use MFA login for both users and the admin page, to ensure that only authorized individuals can access and use the system.
\item As an administrator, I want to be able to create multiple customers, each with their own unique administration flow, so that I can set up and manage multiple organizations within a single solution. % In the context of the Microsoft ecosystem, this would correspond to multiple tenants.
\end{itemize}


\subsection{Should have}
\begin{itemize}
\item As a user, I want to be able to end a meeting early on the screen, so that I can free up the meeting room for another user.
\item As a user, I want to be able to create a meeting room in the administration section and have it automatically created in Exchange Online, so that I can manage the process from a central location and avoid duplication of effort.
\item As a user, I want to be able to customize the application deployment to feature branding such as a custom logo. This enables me to personalize the platform and align it with my company's branding.
\item As a user, I want to be able to send reminders to attendees via email, so that I can ensure that they are aware of the meeting and arrive on time.
\item As a user, I want to be able to view an overview of available meeting rooms, such as a floor plan or other layout, on a large screen, so that I can quickly and easily find an appropriate space for my meeting.
\end{itemize}

\subsection{Could have}
\begin{itemize}
\item As a user, I want to be able to display an info screen if there are no meetings scheduled, until a certain time before the next meeting. This would allow me to use the meeting room for other purposes, such as advertisements or presentations, when it is not being used for meetings.
\end{itemize}

\subsection{Nice to have}
\begin{itemize}
\item As a user, I want to be able to order catering, such as coffee, bread, and lunch, and send an email to the canteen, so that I can easily provide refreshments for my meeting attendees.
\item As a user, I want to be able to order equipment, such as a projector or whiteboard, so that I can make sure that the meeting has the necessary tools to be successful.
\end{itemize}